\problemset{Комбинаторика и теория графов}
\problemset{Индивидуальное домашнее задание №3}	% поменяйте номер ИДЗ

\renewcommand*{\proofname}{Решение}

\begin{problem}
Для заданного на рисунке графа постройте
минимальное остовное дерево, применив алгоритм
Прима (построение начинать с вершины D).
В ответе укажите порядок включения ребер. \\
\usetikzlibrary{graphs,automata,positioning}
\tikz {
	\path
(0, 0) node (A) {A}
(2, 0) node (B) {B}
(4, 0) node (C) {C}
(6, 0) node (D) {D}
(0, -2) node (E) {E}
(2, -2) node (F) {F}
(4, -2) node (G) {G}
(6, -2) node (H) {H}
(0, -4) node (I) {I}
(2, -4) node (J) {J}
(4, -4) node (K) {K}
(6, -4) node (L) {L};
\path[<->]
(A) edge node[sloped,above,node font=\tiny] {4} (B)
(A) edge node[anchor=west,node font=\tiny] {4} (E)
(B) edge node[sloped,above,node font=\tiny] {17} (C)
(B) edge node[sloped,above,node font=\tiny] {19} (E)
(B) edge node[anchor=west,node font=\tiny] {16} (F)
(B) edge node[sloped,above,node font=\tiny] {4} (G)
(C) edge node[sloped,above,node font=\tiny] {7} (D)
(C) edge node[anchor=west,node font=\tiny] {5} (G)
(C) edge node[sloped,above,node font=\tiny] {5} (H)
(D) edge node[anchor=west,node font=\tiny] {3} (H)
(E) edge node[sloped,above,node font=\tiny] {14} (F)
(E) edge node[anchor=west,node font=\tiny] {5} (I)
(E) edge node[sloped,above,node font=\tiny] {9} (J)
(F) edge node[sloped,above,node font=\tiny] {11} (G)
(G) edge node[sloped,above,node font=\tiny] {6} (H)
(G) edge node[sloped,above,node font=\tiny] {19} (J)
(G) edge node[anchor=west,node font=\tiny] {19} (K)
(G) edge node[sloped,above,node font=\tiny] {12} (L)
(I) edge node[sloped,above,node font=\tiny] {6} (J)
(J) edge node[sloped,above,node font=\tiny] {14} (K)
(K) edge node[sloped,above,node font=\tiny] {1} (L)
}
\end{problem}

\begin{proof} $ $ \\
Порядок включения: DH, CH, GC, BG, AB, EA, IE, JI, FG, LG, KL \\
\usetikzlibrary{graphs,automata,positioning}
\tikz {
	\path
(0, 0) node (A) {A}
(2, 0) node (B) {B}
(4, 0) node (C) {C}
(6, 0) node (D) {D}
(0, -2) node (E) {E}
(2, -2) node (F) {F}
(4, -2) node (G) {G}
(6, -2) node (H) {H}
(0, -4) node (I) {I}
(2, -4) node (J) {J}
(4, -4) node (K) {K}
(6, -4) node (L) {L};
\path[<->]
(A) edge node[sloped,above,node font=\tiny] {4} (B)
(A) edge node[anchor=west,node font=\tiny] {4} (E)
(B) edge node[sloped,above,node font=\tiny] {4} (G)
(C) edge node[anchor=west,node font=\tiny] {5} (G)
(C) edge node[sloped,above,node font=\tiny] {5} (H)
(D) edge node[anchor=west,node font=\tiny] {3} (H)
(E) edge node[anchor=west,node font=\tiny] {5} (I)
(F) edge node[sloped,above,node font=\tiny] {11} (G)
(G) edge node[sloped,above,node font=\tiny] {12} (L)
(I) edge node[sloped,above,node font=\tiny] {6} (J)
(K) edge node[sloped,above,node font=\tiny] {1} (L)
}
\end{proof}

\begin{problem}
Для заданного на рисунке графа постройте
минимальное остовное дерево, применив алгоритм
Краскала. \\
\usetikzlibrary{graphs,automata,positioning}
\tikz {
	\path
(0, 0) node (A) {A}
(2, 0) node (B) {B}
(4, 0) node (C) {C}
(6, 0) node (D) {D}
(0, -2) node (E) {E}
(2, -2) node (F) {F}
(4, -2) node (G) {G}
(6, -2) node (H) {H}
(0, -4) node (I) {I}
(2, -4) node (J) {J}
(4, -4) node (K) {K}
(6, -4) node (L) {L};
\path[<->]
(A) edge node[sloped,above,node font=\tiny] {42} (B)
(A) edge node[anchor=west,node font=\tiny] {36} (E)
(B) edge node[sloped,above,node font=\tiny] {22} (C)
(B) edge node[anchor=west,node font=\tiny] {8} (F)
(C) edge node[sloped,above,node font=\tiny] {2} (D)
(C) edge node[sloped,above,node font=\tiny] {15} (F)
(C) edge node[anchor=west,node font=\tiny] {26} (G)
(C) edge node[sloped,above,node font=\tiny] {12} (H)
(D) edge node[anchor=west,node font=\tiny] {39} (H)
(E) edge node[sloped,above,node font=\tiny] {37} (F)
(F) edge node[sloped,above,node font=\tiny] {6} (G)
(F) edge node[sloped,above,node font=\tiny] {13} (I)
(F) edge node[anchor=west,node font=\tiny] {10} (J)
(G) edge node[sloped,above,node font=\tiny] {41} (H)
(G) edge node[sloped,above,node font=\tiny] {31} (J)
(G) edge node[anchor=west,node font=\tiny] {35} (K)
(H) edge node[sloped,above,node font=\tiny] {5} (K)
(H) edge node[anchor=west,node font=\tiny] {23} (L)
(I) edge node[sloped,above,node font=\tiny] {29} (J)
(J) edge node[sloped,above,node font=\tiny] {28} (K)
(K) edge node[sloped,above,node font=\tiny] {32} (L)
}
\end{problem}

\begin{proof} $ $ \\
Порядок включения: CD, HK, FG, BF, FJ, CH, FI, CF, HL, AE, EF \\
\usetikzlibrary{graphs,automata,positioning}
\tikz {
	\path
(0, 0) node (A) {A}
(2, 0) node (B) {B}
(4, 0) node (C) {C}
(6, 0) node (D) {D}
(0, -2) node (E) {E}
(2, -2) node (F) {F}
(4, -2) node (G) {G}
(6, -2) node (H) {H}
(0, -4) node (I) {I}
(2, -4) node (J) {J}
(4, -4) node (K) {K}
(6, -4) node (L) {L};
\path[<->]
(A) edge node[anchor=west,node font=\tiny] {36} (E)
(B) edge node[anchor=west,node font=\tiny] {8} (F)
(C) edge node[sloped,above,node font=\tiny] {2} (D)
(C) edge node[sloped,above,node font=\tiny] {15} (F)
(C) edge node[sloped,above,node font=\tiny] {12} (H)
(E) edge node[sloped,above,node font=\tiny] {37} (F)
(F) edge node[sloped,above,node font=\tiny] {6} (G)
(F) edge node[sloped,above,node font=\tiny] {13} (I)
(F) edge node[anchor=west,node font=\tiny] {10} (J)
(H) edge node[sloped,above,node font=\tiny] {5} (K)
(H) edge node[anchor=west,node font=\tiny] {23} (L)
}
\end{proof}

\begin{problem}
Определите кратчайшие пути от вершины B до всех
вершин графа с помощью алгоритмов Дейкстры и
Форда-Беллмана. В ответе приведите протоколы работы алгоритмов. \\
\usetikzlibrary{graphs,automata,positioning}
\tikz {
	\path
(0, 0) node (A) {A}
(2, 0) node (B) {B}
(4, 0) node (C) {C}
(6, 0) node (D) {D}
(0, -2) node (E) {E}
(2, -2) node (F) {F}
(4, -2) node (G) {G}
(6, -2) node (H) {H}
(0, -4) node (I) {I}
(2, -4) node (J) {J}
(4, -4) node (K) {K}
(6, -4) node (L) {L}
\path
(A) edge[<->] node[sloped,above,node font=\tiny] {8} (B)
(A) edge[->] node[anchor=west,node font=\tiny] {9} (E)
(B) edge[<->] node[sloped,above,node font=\tiny] {3} (C)
(B) edge[<->] node[sloped,above,node font=\tiny] {4} (E)
(D) edge[->] node[sloped,above,node font=\tiny] {7} (C)
(D) edge[<->] node[sloped,above,node font=\tiny] {5} (G)
(E) edge[<->] node[anchor=west,node font=\tiny] {2} (I)
(F) edge[->] node[anchor=west,node font=\tiny] {8} (B)
(F) edge[->] node[sloped,above,node font=\tiny] {10} (C)
(F) edge[->] node[sloped,above,node font=\tiny] {8} (E)
(F) edge[->] node[sloped,above,node font=\tiny] {10} (G)
(F) edge[<->] node[anchor=west,node font=\tiny] {9} (J)
(F) edge[->] node[sloped,above,node font=\tiny] {10} (K)
(G) edge[<->] node[sloped,above,node font=\tiny] {5} (H)
(H) edge[->] node[anchor=west,node font=\tiny] {10} (D)
(H) edge[<->] node[sloped,above,node font=\tiny] {7} (K)
(H) edge[->] node[anchor=west,node font=\tiny] {7} (L)
(I) edge[->] node[sloped,above,node font=\tiny] {9} (J)
(J) edge[<->] node[sloped,above,node font=\tiny] {10} (K)
(K) edge[->] node[anchor=west,node font=\tiny] {5} (G)
(K) edge[<->] node[sloped,above,node font=\tiny] {2} (L)
}
\end{problem}

\begin{proof} $ $ \\
Алгоритм Дейкстры: \\
\begin{tabular}{c|c|c|c|c|c|c|c|c|c|c|c}
    A & B & C & D & E & F & G & H & I & J & K & L \\
    \hline
    8 & 0 & 3 & $\infty$ & 4 & $\infty$ & $\infty$ & $\infty$ & $\infty$ & $\infty$ & $\infty$ & $\infty$ \\
8 & $ $ & 3 & $\infty$ & 4 & $\infty$ & $\infty$ & $\infty$ & $\infty$ & $\infty$ & $\infty$ & $\infty$ \\
8 & $ $ & $ $ & $\infty$ & 4 & $\infty$ & $\infty$ & $\infty$ & 6 & $\infty$ & $\infty$ & $\infty$ \\
8 & $ $ & $ $ & $\infty$ & $ $ & $\infty$ & $\infty$ & $\infty$ & 6 & 15 & $\infty$ & $\infty$ \\
8 & $ $ & $ $ & $\infty$ & $ $ & 24 & $\infty$ & $\infty$ & $ $ & 15 & 25 & $\infty$ \\
8 & $ $ & $ $ & $\infty$ & $ $ & 24 & 34 & $\infty$ & $ $ & $ $ & 25 & $\infty$ \\
8 & $ $ & $ $ & 39 & $ $ & $ $ & 34 & 39 & $ $ & $ $ & 25 & $\infty$ \\
8 & $ $ & $ $ & 39 & $ $ & $ $ & $ $ & 39 & $ $ & $ $ & 25 & $\infty$ \\
8 & $ $ & $ $ & $ $ & $ $ & $ $ & $ $ & 39 & $ $ & $ $ & 25 & $\infty$ \\
$ $ & $ $ & $ $ & $ $ & $ $ & $ $ & 30 & 32 & $ $ & $ $ & 25 & 27 \\
$ $ & $ $ & $ $ & $ $ & $ $ & $ $ & 30 & 32 & $ $ & $ $ & $ $ & 27 \\
$ $ & $ $ & $ $ & 35 & $ $ & $ $ & 30 & 32 & $ $ & $ $ & $ $ & $ $ \\
$ $ & $ $ & $ $ & 35 & $ $ & $ $ & $ $ & 32 & $ $ & $ $ & $ $ & $ $ \\
$ $ & $ $ & $ $ & $ $ & $ $ & $ $ & $ $ & 32 & $ $ & $ $ & $ $ & $ $ \\
\hline
8 & 0 & 3 & 35 & 4 & 24 & 30 & 32 & 6 & 15 & 25 & 27 \\
\end{tabular}
\\\\
Алгоритм Форда-Беллмана:\\
\begin{tabular}{c|c|c|c|c|c|c|c|c|c|c|c}
    A & B & C & D & E & F & G & H & I & J & K & L \\
    \hline
    $\infty$ & 0 & $\infty$ & $\infty$ & $\infty$ & $\infty$ & $\infty$ & $\infty$ & $\infty$ & $\infty$ & $\infty$ & $\infty$ \\
8 & 0 & 3 & $\infty$ & 4 & 24 & 30 & 32 & 6 & 15 & 25 & 27 \\
8 & 0 & 3 & 35 & 4 & 24 & 30 & 32 & 6 & 15 & 25 & 27 \\
8 & 0 & 3 & 35 & 4 & 24 & 30 & 32 & 6 & 15 & 25 & 27 \\
8 & 0 & 3 & 35 & 4 & 24 & 30 & 32 & 6 & 15 & 25 & 27 \\
8 & 0 & 3 & 35 & 4 & 24 & 30 & 32 & 6 & 15 & 25 & 27 \\
8 & 0 & 3 & 35 & 4 & 24 & 30 & 32 & 6 & 15 & 25 & 27 \\
8 & 0 & 3 & 35 & 4 & 24 & 30 & 32 & 6 & 15 & 25 & 27 \\
8 & 0 & 3 & 35 & 4 & 24 & 30 & 32 & 6 & 15 & 25 & 27 \\
8 & 0 & 3 & 35 & 4 & 24 & 30 & 32 & 6 & 15 & 25 & 27 \\
8 & 0 & 3 & 35 & 4 & 24 & 30 & 32 & 6 & 15 & 25 & 27 \\
8 & 0 & 3 & 35 & 4 & 24 & 30 & 32 & 6 & 15 & 25 & 27 \\
\hline
8 & 0 & 3 & 35 & 4 & 24 & 30 & 32 & 6 & 15 & 25 & 27 \\
\end{tabular}
\end{proof}

\begin{problem}
С помощью алгоритма Флойда определите кратчайшие пути между всеми парами вершин графа, а также сами пути. В решении представить все матрицы, соответствующие последователному расширению множества промежуточных вершин. Смените знак весам двух любых ребер так, чтобы в графе не возникало циклов отрицательного суммарного веса, и примените к нему алгоритм Джонсона. \\ 
\usetikzlibrary{graphs,automata,positioning}
\tikz {
	\path
(2, 0) node (A) {A}
(0, -2) node (B) {B}
(6, -2) node (C) {C}
(2, -4) node (D) {D}
(6, -4) node (E) {E};
\path
(A) edge[->] node[sloped,above,node font=\tiny] {13} (B)
(A) edge[->] node[pos=0.4,anchor=west,node font=\tiny] {9} (D)
(A) edge[->] node[pos=0.4,anchor=west,node font=\tiny] {12} (E)
(B) edge[->] node[sloped,above,node font=\tiny] {4} (C)
(B) edge[->] node[sloped,above,node font=\tiny] {15} (D)
(C) edge[->] node[sloped,above,node font=\tiny] {15} (A)
(C) edge[->] node[anchor=west,node font=\tiny] {5} (E)
(D) edge[->] node[sloped,above,node font=\tiny] {9} (C)
(D) edge[->] node[sloped,above,node font=\tiny] {7} (E)
(E) edge[->] node[sloped,above,node font=\tiny] {6} (B)
}
\end{problem}

\begin{proof} $ $ \\
Алгоритм Флойда-Уоршелла: \\
\begin{tabular}{c|c|c|c|c|c}
$ $ & A & B & C & D & E \\
\hline
A & 0 & 13 & $\infty$ & 9 & 12 \\
\hline
B & $\infty$ & 0 & 4 & 15 & $\infty$ \\
\hline
C & 15 & $\infty$ & 0 & $\infty$ & 5 \\
\hline
D & $\infty$ & $\infty$ & 9 & 0 & 7 \\
\hline
E & $\infty$ & 6 & $\infty$ & $\infty$ & 0 \\
\end{tabular}
\quad
\begin{tabular}{c|>{\columncolor{gray!30}}c|c|c|c|c}
$ $ & A & B & C & D & E \\
\hline
\rowcolor{gray!30}
A & 0 & 13 & $\infty$ & 9 & 12 \\
\hline
B & $\infty$ & 0 & 4 & 15 & $\infty$ \\
\hline
C & 15 & 28 & 0 & 24 & 5 \\
\hline
D & $\infty$ & $\infty$ & 9 & 0 & 7 \\
\hline
E & $\infty$ & 6 & $\infty$ & $\infty$ & 0 \\
\end{tabular}
\quad
\begin{tabular}{c|c|>{\columncolor{gray!30}}c|c|c|c}
$ $ & A & B & C & D & E \\
\hline
A & 0 & 13 & 17 & 9 & 12 \\
\hline
\rowcolor{gray!30}
B & $\infty$ & 0 & 4 & 15 & $\infty$ \\
\hline
C & 15 & 28 & 0 & 24 & 5 \\
\hline
D & $\infty$ & $\infty$ & 9 & 0 & 7 \\
\hline
E & $\infty$ & 6 & 10 & 21 & 0 \\
\end{tabular}
\\\\\\
\begin{tabular}{c|c|c|>{\columncolor{gray!30}}c|c|c}
$ $ & A & B & C & D & E \\
\hline
A & 0 & 13 & 17 & 9 & 12 \\
\hline
B & 19 & 0 & 4 & 15 & 9 \\
\hline
\rowcolor{gray!30}
C & 15 & 28 & 0 & 24 & 5 \\
\hline
D & 24 & 37 & 9 & 0 & 7 \\
\hline
E & 25 & 6 & 10 & 21 & 0 \\
\end{tabular}
\quad
\begin{tabular}{c|c|c|c|>{\columncolor{gray!30}}c|c}
$ $ & A & B & C & D & E \\
\hline
A & 0 & 13 & 17 & 9 & 12 \\
\hline
B & 19 & 0 & 4 & 15 & 9 \\
\hline
C & 15 & 28 & 0 & 24 & 5 \\
\hline
\rowcolor{gray!30}
D & 24 & 37 & 9 & 0 & 7 \\
\hline
E & 25 & 6 & 10 & 21 & 0 \\
\end{tabular}
\quad
\begin{tabular}{c|c|c|c|c|>{\columncolor{gray!30}}c}
$ $ & A & B & C & D & E \\
\hline
A & 0 & 13 & 17 & 9 & 12 \\
\hline
B & 19 & 0 & 4 & 15 & 9 \\
\hline
C & 15 & 11 & 0 & 24 & 5 \\
\hline
D & 24 & 13 & 9 & 0 & 7 \\
\hline
\rowcolor{gray!30}
E & 25 & 6 & 10 & 21 & 0 \\
\end{tabular} \\\\
Сменим знак рёбрам CE и AD: \\
\usetikzlibrary{graphs,automata,positioning}
\tikz {
	\path
(2, 0) node (A) {A}
(0, -2) node (B) {B}
(6, -2) node (C) {C}
(2, -4) node (D) {D}
(6, -4) node (E) {E};
\path
(A) edge[->] node[sloped,above,node font=\tiny] {13} (B)
(A) edge[->] node[pos=0.4,anchor=west,node font=\tiny] {-9} (D)
(A) edge[->] node[pos=0.4,anchor=west,node font=\tiny] {12} (E)
(B) edge[->] node[sloped,above,node font=\tiny] {4} (C)
(B) edge[->] node[sloped,above,node font=\tiny] {15} (D)
(C) edge[->] node[sloped,above,node font=\tiny] {15} (A)
(C) edge[->] node[anchor=west,node font=\tiny] {-5} (E)
(D) edge[->] node[sloped,above,node font=\tiny] {9} (C)
(D) edge[->] node[sloped,above,node font=\tiny] {7} (E)
(E) edge[->] node[sloped,above,node font=\tiny] {6} (B)
} \\
Применим алгоритм Джонсона:\\
Добавим вершину $s$ и применим алгоритм Беллмана-Форда: \\ 
\usetikzlibrary{graphs,automata,positioning}
\tikz {
	\path
(4, 0) node (A) {A}
(0, -2) node (s) {s}
(2, -2) node (B) {B}
(8, -2) node (C) {C}
(4, -4) node (D) {D}
(8, -4) node (E) {E};
\path
(A) edge[->] node[sloped,above,node font=\tiny] {13} (B)
(A) edge[->] node[pos=0.4, anchor=west,node font=\tiny] {-9} (D)
(A) edge[->] node[pos=0.4,sloped,above,node font=\tiny] {12} (E)
(B) edge[->] node[sloped,above,node font=\tiny] {4} (C)
(B) edge[->] node[sloped,above,node font=\tiny] {15} (D)
(C) edge[->] node[sloped,above,node font=\tiny] {15} (A)
(C) edge[->] node[anchor=west,node font=\tiny] {-5} (E)
(D) edge[->] node[sloped,above,node font=\tiny] {9} (C)
(D) edge[->] node[sloped,above,node font=\tiny] {7} (E)
(E) edge[->] node[sloped,above,node font=\tiny] {6} (B)
(s) edge[->, bend right =-20] node[sloped,above,node font=\tiny] {0} (A)
(s) edge[->] node[sloped,above,node font=\tiny] {0} (B)
(s) edge[->, bend right = -80] node[sloped,above,node font=\tiny] {0} (C)
(s) edge[->, bend left =-20] node[sloped,above,node font=\tiny] {0} (D)
(s) edge[->, bend left =-50] node[sloped,above,node font=\tiny] {0} (E)
} \\
\begin{tabular}{c|c|c|c|c|c}
     $ $ & A & B & C & D & E \\
     \hline
     s & 0 & 0 & 0 & -9 & -5 \\
\end{tabular} \\\\
Изменим веса рёбер, чтобы исключить отрицательные: \\
\usetikzlibrary{graphs,automata,positioning}
\tikz {
	\path
(2, 0) node (A) {A}
(0, -2) node (B) {B}
(6, -2) node (C) {C}
(2, -4) node (D) {D}
(6, -4) node (E) {E};
\path
(A) edge[->] node[sloped,above,node font=\tiny] {13} (B)
(A) edge[->] node[pos=0.4,anchor=west,node font=\tiny] {0} (D)
(A) edge[->] node[pos=0.4,sloped,above,node font=\tiny] {17} (E)
(B) edge[->] node[sloped,above,node font=\tiny] {4} (C)
(B) edge[->] node[sloped,above,node font=\tiny] {24} (D)
(C) edge[->] node[sloped,above,node font=\tiny] {15} (A)
(C) edge[->] node[anchor=west,node font=\tiny] {0} (E)
(D) edge[->] node[sloped,above,node font=\tiny] {0} (C)
(D) edge[->] node[sloped,above,node font=\tiny] {3} (E)
(E) edge[->] node[sloped,above,node font=\tiny] {1} (B)
} \\
Применим алгоритм Дейкстры и восстановим веса рёбер:\\\\
\begin{tabular}{c|c|c|c|c|c}
$ $ & A & B & C & D & E \\
\hline
A & 0 & 1 & 0 & 0 & 0\\
\hline
B & 19 & 0 & 4 & 19 & 4\\
\hline
C & 15 & 1 & 0 & 15 & 0\\
\hline 
D & 15 & 1 & 0 & 0 & 0\\
\hline
E & 20 & 1 & 5 & 20 & 0\\
\end{tabular}
\quad $\Rightarrow$ \quad
\begin{tabular}{c|c|c|c|c|c}
$ $ & A & B & C & D & E \\
\hline
A & 0 & 1 & 0 & -9 & -5\\
\hline
B & 19 & 0 & 4 & 10 & -1\\
\hline
C & 15 & 1 & 0 & 6 & -5\\
\hline
D & 24 & 10 & 9 & 0 & 4\\
\hline
E & 25 & 6 & 10 & 16 & 0\\
\end{tabular}
\end{proof}