\problemset{Комбинаторика и теория графов}
\problemset{Индивидуальное домашнее задание №2}	% поменяйте номер ИДЗ

\renewcommand*{\proofname}{Решение}

\begin{problem}
Является ли граф а) эйлеровым, полуэйлеровым?
б) гамильтоновым, полугамильтоновым? в) двудольным? г)
вершинно-двусвязным; д) рёберно-двусвязным е) постройте
дерево блоков и точек сочленения. \\
\usetikzlibrary{graphs,automata,positioning}
\tikz {
	\path
(2, 0) node (A) {A}
(4, 0) node (B) {B}
(0, -2) node (C) {C}
(2, -2) node (D) {D}
(4, -2) node (E) {E}
(6, -2) node (F) {F}
(0, -4) node (G) {G}
(2, -4) node (H) {H}
(4, -4) node (I) {I}
(6, -4) node (J) {J}
(2, -6) node (K) {K}
(4, -6) node (L) {L};
\draw (A) -- (B);
\draw (A) -- (E);
\draw (B) -- (D);
\draw (C) -- (D);
\draw (C) -- (G);
\draw (D) -- (E);
\draw (D) -- (H);
\draw (E) -- (I);
\draw (F) -- (I);
\draw (F) -- (J);
\draw (G) -- (D);
\draw (H) -- (I);
\draw (I) -- (L);
\draw (K) -- (H);
\draw (K) -- (L);
\draw (L) -- (L);
}
\end{problem}

\begin{proof} $ $
    \begin{enumerate}[label=\asbuk*), ref=\asbuk*]
        \item \textbf{эйлеровым, полуэйлеровым?\\}
        Граф не является ни эйлеровым, ни полуэйлеровым, так как в графе есть вершины с нечётными степенями, и их больше двух (D, E, H, J).
        \item \textbf{гамильтоновым, полугамильтоновым?\\}
        В графе есть лист J, следовательно, он не является гамильтоновым. Так как между шарнирами D и I нет простого пусти, содержащего вершины A, B, E, H, K, L, то граф не является полугамильтоновым.
        \item \textbf{двудольным?\\}
\usetikzlibrary{graphs,automata,positioning}
\tikz {
	\path
(1, 0) node (A) {A$^*$}
(2, 0) node (B) {B$^\#$}
(0, -1) node (C) {C$^\#$}
(1, -1) node (D) {D$^*$}
(2, -1) node (E) {E$^\#$}
(3, -1) node (F) {F$^\#$}
(0, -2) node (G) {G$^\#$}
(1, -2) node (H) {H$^\#$}
(2, -2) node (I) {I$^*$}
(3, -2) node (J) {J$^*$}
(1, -3) node (K) {K$^*$}
(2, -3) node (L) {L$^\#$};
\draw (A) -- (B);
\draw (A) -- (E);
\draw (B) -- (D);
\draw (C) -- (D);
\draw (C) -- (G);
\draw (D) -- (E);
\draw (D) -- (H);
\draw (E) -- (I);
\draw (F) -- (I);
\draw (F) -- (J);
\draw (G) -- (D);
\draw (H) -- (I);
\draw (I) -- (L);
\draw (K) -- (H);
\draw (K) -- (L);
\draw (L) -- (L);
}\\
    Так как при раскрашивании смежные вершины C и G оказались одного цвета, граф не является двудольным.
    \item \textbf{вершинно-двусвязным?\\}
    Так как в графе присутствуют шарниры (D, F, I), то граф не может быть вершинно-двусвязным. 
    \item \textbf{рёберно-двусвязным?\\}
    Так как в графе есть мосты (FI, FJ), то граф не является рёберно-двусвязным.
    \item \textbf{постройте
дерево блоков и точек сочленения.\\}
\usetikzlibrary{graphs,automata,positioning}
\tikz {
	\path
(0, 0) node[white, circle,fill=black] (CG) {CG}
(2, 0) node[circle,fill=red] (D) {D}
(10, 0) node[circle,fill=red] (F) {F}
(4, -2) node[white, circle,fill=black] (ABEHKL) {ABEHKL}
(8, -2) node[white, circle,fill=black] (IF) {IF}
(10, -2) node[white, circle,fill=black] (J) {J}
(6, -4) node[circle,fill=red] (I) {I};
\draw (CG) -- (D);
\draw (D) -- (ABEHKL);
\draw (ABEHKL) -- (I);
\draw (F) -- (IF);
\draw (F) -- (J);
\draw (IF) -- (I);
}
    \end{enumerate}
\end{proof}

\begin{problem}
\end{problem}

\begin{proof} $ $
\end{proof}

\begin{problem}
\end{problem}

\begin{proof} $ $
\end{proof}

\begin{problem}
а) Постройте код Прюфера для данного дерева:\\
\usetikzlibrary{graphs,automata,positioning}
\tikz {
	\path
(0, 0) node (1) {1}
(2, 0) node (2) {2}
(4, 0) node (3) {3}
(6, 0) node (4) {4}
(8, 0) node (5) {5}
(10, 0) node (6) {6}
(0, -2) node (7) {7}
(2, -2) node (8) {8}
(4, -2) node (9) {9}
(6, -2) node (10) {10}
(8, -2) node (11) {11};
\draw (1) -- (2);
\draw (2) -- (3);
\draw (3) -- (7);
\draw (4) -- (7);
\draw (5) -- (7);
\draw (6) -- (11);
\draw (8) -- (5);
\draw (9) -- (10);
\draw (10) -- (11);
\draw (11) -- (5);
} \\
б) Постройте дерево по коду Прюфера: 3 9 4 9 10 1 1 9 10.
\end{problem}

\begin{proof} $ $
    \begin{enumerate}[label=\asbuk*), ref=\asbuk*]
        \item \textbf{Построим код Прюфера\\}
        2 3 7 7 11 5 5 11 10 11
        \item \textbf{Построим дерево по коду\\}
        Код: 3 9 4 9 10 1 1 9 10 \\
        Порядок: 2 3 5 4 6 7 8 1 9\\
        \usetikzlibrary{graphs,automata,positioning}
\tikz {
	\path
(0, 0) node (2) {2}
(2, 0) node (3) {3}
(4, 0) node (9) {9}
(6, 0) node (10) {10}
(8, 0) node (6) {6}
(0, -2) node (5) {5}
(2, -2) node (4) {4}
(4, -2) node (7) {7}
(6, -2) node (1) {1}
(8, -2) node (8) {8};
\draw (1) -- (7);
\draw (1) -- (8);
\draw (2) -- (3);
\draw (3) -- (9);
\draw (4) -- (5);
\draw (4) -- (9);
\draw (6) -- (10);
\draw (9) -- (1);
\draw (9) -- (10);
}
    \end{enumerate}
\end{proof}

\begin{problem}
\end{problem}

\begin{proof} $ $
\end{proof}

\begin{problem}
\end{problem}

\begin{proof} $ $
\end{proof}

\begin{problem}
\end{problem}

\begin{proof} $ $
\end{proof}

\begin{problem}
Найдите наибольшее паросочетание в двудольном графе, заданном набором рёбер: (a, $\beta$) (a, $\eta$) (b, $\beta$) (b, $\varepsilon$) (c, $\alpha$)
(c, $\zeta$) (d, $\varepsilon$) (e, $\zeta$) (e, $\eta$) (e, $\theta$) (f, $\gamma$) (f, $\delta$) (f, $\eta$) (g, $\zeta$) (h, $\beta$) (h, $\gamma$)
\end{problem}

\begin{proof} $\\$
\usetikzlibrary{graphs,automata,positioning}
\tikz {
	\path
(0, -7) node (S) {S}
(2, 0) node (a) {a}
(8, 0) node (1) {$\alpha$}
(2, -2) node (b) {b}
(8, -2) node (2) {$\beta$}
(2, -4) node (c) {c}
(8, -4) node (3) {$\eta$}
(2, -6) node (d) {d}
(8, -6) node (4) {$\varepsilon$}
(2, -8) node (e) {e}
(8, -8) node (5) {$\zeta$}
(2, -10) node (f) {f}
(8, -10) node (6) {$\theta$}
(2, -12) node (g) {g}
(8, -12) node (7) {$\gamma$}
(2, -14) node (h) {h}
(8, -14) node (8) {$\delta$}
(10, -7) node (F) {F};
\draw[->] (a) -- (2);
\draw[<-][very thick] (a) -- (3);
\draw[<-][very thick] (b) -- (2);
\draw[->] (b) -- (4);
\draw[<-][very thick] (c) -- (1);
\draw[->] (c) -- (5);
\draw[<-][very thick] (d) -- (4);
\draw[->] (e) -- (5);
\draw[->] (e) -- (3);
\draw[<-][very thick] (e) -- (6);
\draw[->] (f) -- (7);
\draw[<-][very thick] (f) -- (8);
\draw[->] (f) -- (3);
\draw[<-][very thick] (g) -- (5);
\draw[->] (h) -- (2);
\draw[<-][very thick] (h) -- (7);
}
\\ Ответ: (a, $\eta$) (b, $\beta$) (c, $\alpha$)
(d, $\varepsilon$) (e, $\theta$) (f, $\delta$)  (g, $\zeta$) (h, $\gamma$)
\end{proof}