\documentclass{amsart}

\usepackage[utf8]{inputenc}
\usepackage[T2A]{fontenc}
\usepackage[english,russian]{babel}
\usepackage{amsthm,amsmath,amsfonts,amssymb}
\usepackage{fullpage}
\usepackage{eufrak}

%%% Дополнительная работа с математикой
\usepackage{amsfonts,amssymb,amsthm,mathtools} % AMS
\usepackage{amsmath}
\usepackage{icomma}

%% Графы
\usepackage{tikz-cd}  
\usepackage{tikz}
\usetikzlibrary{graphs,automata,positioning}

%% Шрифты
\usepackage{euscript}	 % Шрифт Евклид
\usepackage{mathrsfs} % Красивый матшрифт

\usepackage{float}

%% Свои команды
\DeclareMathOperator{\sgn}{\mathop{sgn}}	% сигнум
\DeclareMathOperator{\cov}{\mathop{cov}}	% ковариация
\DeclareMathOperator{\lb}{\mathop{lb}}	% бинарный логарифм (логарифм по основанию 2)
\DeclareMathOperator{\supp}{\mathop{supp}}	% носитель

\renewcommand{\Im}{\mathop{\mathrm{Im}}\nolimits}	% мнимая часть
\renewcommand{\Re}{\mathop{\mathrm{Re}}\nolimits}	% вещественная часть

\renewcommand{\Prob}{\mathbb P}	% вероятность
\newcommand{\Expect}{\mathbb E}	% математическое ожидание
\renewcommand{\Variance}{\mathbb D}	% дисперсия
\newcommand{\Entropy}{\mathbb H}	% энтропия

%% Перенос знаков в формулах (по Львовскому)
\newcommand*{\hm}[1]{#1\nobreak\discretionary{}
	{\hbox{$\mathsurround=0pt #1$}}{}}

%%% Работа с картинками
\usepackage{graphicx}  % Для вставки рисунков
\graphicspath{{images/}{images2/}}  % папки с картинками
\setlength\fboxsep{3pt} % Отступ рамки \fbox{} от рисунка
\setlength\fboxrule{1pt} % Толщина линий рамки \fbox{}
\usepackage{wrapfig} % Обтекание рисунков и таблиц текстом
\RequirePackage{caption}
\DeclareCaptionLabelSeparator{defffis}{ -- }
\captionsetup{justification=centering,labelsep=defffis}

\renewcommand{\qedsymbol}{}

%%% Работа с таблицами
\usepackage{array,tabularx,tabulary,booktabs} % Дополнительная работа с таблицами
\usepackage{longtable}  % Длинные таблицы
\usepackage{multirow} % Слияние строк в таблице

\newtheorem{problem}{Задание}

\begin{document}
	
	\newcommand{\problemset}[1]{
		\begin{center}
			\Large #1
		\end{center}
	}
	
	\begin{tabbing}
	\hspace{11cm} \= Студент: \= Коротков Фёдор \\ % не забудьте исправить, студент Вы или студентка :)
																									% (а то некоторые забывают)
	\> Группа: \> 2362 \\  % Здесь меняете № группы
	\> Вариант: \> QG \\    % А здесь меняете № варианта
	\> Дата: \> \today     % А вот здесь ничего не меняем!!!
\end{tabbing}
\hrule
\vspace{1cm}  % в данном файле меняем только Пол, Фамилию Имя, № группы и № варианта
	\problemset{Комбинаторика и теория графов}
\problemset{Индивидуальное домашнее задание №1}	% поменяйте номер ИДЗ

\renewcommand*{\proofname}{Решение}
%%%%%%%%%%%%%% ЗАДАНИЕ №1 %%%%%%%%%%%%%%
%% Условие задания №1
\begin{problem}
	Дано множество М = $\{19, 33, 69, 72, 77, 91, 96, 97\}$
    и следующие бинарные отношения на нем:
    \begin{itemize}
    
    \item $F_1(x,y) = 1 \Leftrightarrow \exists z \in M : (x - z)(y - z) < 0;$

    \item $F_2(x, y) = 1 \Leftrightarrow x \geq y$ поразрядно;

    \item $F_3(x, y) = 1 \Leftrightarrow [\frac{x}{5}] = [\frac{y}{5}]$;

    \item $F_4(x,y) = 1 \Leftrightarrow x^2 - y^3$ нечетно;

    \item $F_5(x, y) = 1 \Leftrightarrow |x-y| < 10$.
    \end{itemize}
    Для каждого из отношений:

    \begin{enumerate}

    \item[1.] Проверить, является ли бинарное отношение (далее -  б.о.) - рефлексивным, арефлексивным, симметричным, антисимметричным, асимметричным, транзитивным.
    \item[2.] Построить матрицы и графы этих б.о.

    \item[3.] Определить, являются ли эти б.о. отношениями эквивалентности, частичного порядка, линейного порядка, строгого порядка).

    \item[4.] Для отношений эквивалентности построить классы эквивалентности.

    \item[5.] Для отношений частичного порядка применить алгоритм топологической сортировки и получить отношение линейного порядка.

    \item[6.] Для нетранзитивных отношений построить транзитивное замыкание, используя алгоритм Уоршелла.
    \end{enumerate}
\end{problem}

%% Решение задания №1
\pagebreak
\begin{proof} \textbf{Бинарное отношение $F_1$}\\
Отношение $F_1$ можно переформулирвоать, как $\exists z \in M: x < z < y \text{ или } y < z < x$
\\ \\
Построим матрицу смежности для б.о. $F_1$
$$ \left( \begin{array}{cccccccc}
		   0 &0 &1 &1 &1 &1 &1 &1 
        \\0 &0 &0 &1 &1 &1 &1 &1
        \\1 &0 &0 &0 &1 &1 &1 &1
        \\1 &1 &0 &0 &0 &1 &1 &1
        \\1 &1 &1 &0 &0 &0 &1 &1
        \\1 &1 &1 &1 &0 &0 &0 &1
        \\1 &1 &1 &1 &1 &0 &0 &0
        \\1 &1 &1 &1 &1 &1 &0 &0 \end{array} \right) $$\\
Из матрицы смежности видно, что бинарное отношение $F_1$ является арефлексивным, т.к. элементы матрицы смежности на главной диагонали не равны единице, и симметричным, т.к. относительно главной диагонали матрица зеркальна.
\\ \\
Построим граф для б.о. $F_1$:

\tikz {
        \path
(4.0, 8.0) node[state] (1) {19}
(6.83, 6.83) node[state] (2) {33}
(8.0, 4.0) node[state] (3) {69}
(6.83, 1.17) node[state] (4) {72}
(4.0, 0.0) node[state] (5) {77}
(1.17, 1.17) node[state] (6) {91}
(0.0, 4.0) node[state] (7) {96}
(1.17, 6.83) node[state] (8) {97};
\draw (1) -- (3);
\draw (1) -- (4);
\draw (1) -- (5);
\draw (1) -- (6);
\draw (1) -- (7);
\draw (1) -- (8);
\draw (2) -- (4);
\draw (2) -- (5);
\draw (2) -- (6);
\draw (2) -- (7);
\draw (2) -- (8);
\draw (3) -- (5);
\draw (3) -- (6);
\draw (3) -- (7);
\draw (3) -- (8);
\draw (4) -- (6);
\draw (4) -- (7);
\draw (4) -- (8);
\draw (5) -- (7);
\draw (5) -- (8);
\draw (6) -- (8);
}\\
Бинарное отношение не является транзитивным, т.к., например, между вершинами 96 и 97 есть путь длины 2, но нет пути длины 1.
\ \
Отношение $F_1$:
\begin{itemize}
    \item Арефлексивное
    \item Симметричное
    \item Не транзитивное 
\end{itemize}
\ \
Для построения транзитивного замыкания применим алгоритм Уоршелла:
$$ \left( \begin{array}{cccccccc}
		   0 &0 &1 &1 &1 &1 &1 &1 
        \\0 &0 &0 &1 &1 &1 &1 &1
        \\1 &0 &0 &0 &1 &1 &1 &1
        \\1 &1 &0 &0 &0 &1 &1 &1
        \\1 &1 &1 &0 &0 &0 &1 &1
        \\1 &1 &1 &1 &0 &0 &0 &1
        \\1 &1 &1 &1 &1 &0 &0 &0
        \\1 &1 &1 &1 &1 &1 &0 &0 \end{array} \right) \Rightarrow
    \left( \begin{array}{cccccccc}
		   0 &0 &1 &1 &1 &1 &1 &1 
        \\0 &0 &0 &1 &1 &1 &1 &1
        \\1 &0 &\textcolor{red}{1} &\textcolor{red}{1} &1 &1 &1 &1
        \\1 &1 &\textcolor{red}{1} &\textcolor{red}{1} &\textcolor{red}{1} &1 &1 &1
        \\1 &1 &1 &\textcolor{red}{1} &\textcolor{red}{1} &\textcolor{red}{1} &1 &1
        \\1 &1 &1 &1 &\textcolor{red}{1} &\textcolor{red}{1} &\textcolor{red}{1} &1
        \\1 &1 &1 &1 &1 &\textcolor{red}{1} &\textcolor{red}{1} &\textcolor{red}{1}
        \\1 &1 &1 &1 &1 &1 &\textcolor{red}{1} &\textcolor{red}{1} \end{array} \right) \Rightarrow
    \left( \begin{array}{cccccccc}
		   \textcolor{red}{1} &\textcolor{red}{1} &1 &1 &1 &1 &1 &1 
        \\\textcolor{red}{1} &\textcolor{red}{1} &\textcolor{red}{1} &1 &1 &1 &1 &1
        \\1 &\textcolor{red}{1} &1 &1 &1 &1 &1 &1
        \\1 &1 &1 &1 &1 &1 &1 &1
        \\1 &1 &1 &1 &1 &1 &1 &1
        \\1 &1 &1 &1 &1 &1 &1 &1
        \\1 &1 &1 &1 &1 &1 &1 &1
        \\1 &1 &1 &1 &1 &1 &1 &1 \end{array} \right)    
        $$
\end{proof}

\pagebreak
\begin{proof} \textbf{Бинарное отношение $F_2$}\\
Построим матрицу смежности для б.о. $F_2$
$$ \left( \begin{array}{cccccccc}
		   1 &0 &0 &0 &0 &0 &0 &0 
        \\0 &1 &0 &0 &0 &0 &0 &0
        \\1 &1 &1 &0 &0 &0 &0 &0
        \\0 &0 &0 &1 &0 &0 &0 &0
        \\0 &1 &0 &1 &1 &0 &0 &0
        \\0 &0 &0 &0 &0 &1 &0 &0
        \\0 &1 &0 &1 &0 &1 &1 &0
        \\0 &1 &0 &1 &1 &1 &1 &1 \end{array} \right) $$\\
Бинарное отношение $F_2$ рефлексивное, т.к все элементы матрицы смежности на главной диагонали равны единице, и антисимметричное, т.к. выше главное диагонали в матрице находятся только 0.

\tikz {
	\path
(4.0, 8.0) node[state] (1) {19}
(6.83, 6.83) node[state] (2) {33}
(8.0, 4.0) node[state] (3) {69}
(6.83, 1.17) node[state] (4) {72}
(4.0, 0.0) node[state] (5) {77}
(1.17, 1.17) node[state] (6) {91}
(0.0, 4.0) node[state] (7) {96}
(1.17, 6.83) node[state] (8) {97};
\draw[->] (1) to [out=45.0,in=90.0,looseness=5] (1);
\draw[->] (2) to [out=0.0,in=45.0,looseness=5] (2);
\draw[->] (3) -- (1);
\draw[->] (3) -- (2);
\draw[->] (3) to [out=-45.0,in=0.0,looseness=5] (3);
\draw[->] (4) to [out=-90.0,in=-45.0,looseness=5] (4);
\draw[->] (5) -- (2);
\draw[->] (5) -- (4);
\draw[->] (5) to [out=-135.0,in=-90.0,looseness=5] (5);
\draw[->] (6) to [out=-180.0,in=-135.0,looseness=5] (6);
\draw[->] (7) -- (2);
\draw[->] (7) -- (4);
\draw[->] (7) -- (6);
\draw[->] (7) to [out=-225.0,in=-180.0,looseness=5] (7);
\draw[->] (8) -- (2);
\draw[->] (8) -- (4);
\draw[->] (8) -- (5);
\draw[->] (8) -- (6);
\draw[->] (8) -- (7);
\draw[->] (8) to [out=-270.0,in=-225.0,looseness=5] (8);
}\\
Бинарное отношение $F_2$ транзитивно, т.к. для любых вершин, между котрыми есть путь длины 2 найдется и путь длины 1.
\\ \\
Отношение $F_2$:
\begin{itemize}
    \item Рефлексивное
    \item Антисимметричное
    \item Транзитивное
\end{itemize}
Соответственно, оно является отношением частичного порядка (не линейный порядок, т.к. на графе не между всеми вершинами есть ребро).
\\ \\
Граф после применения топологической сортировки для дополнения отношения $F_2$ до отношения линейного порядка будет выглядеть следующим образом (если в качестве начальной вершины каждый раз принимать вершину с наименьшим значением из множества ещё не пройденных вершин, то с возрастанием номера вершины, её номер будет наоборот убывать):

\tikz {
	\path
(4.0, 8.0) node[state] (1) {19}
(6.83, 6.83) node[state] (2) {33}
(8.0, 4.0) node[state] (3) {69}
(6.83, 1.17) node[state] (4) {72}
(4.0, 0.0) node[state] (5) {77}
(1.17, 1.17) node[state] (6) {91}
(0.0, 4.0) node[state] (7) {96}
(1.17, 6.83) node[state] (8) {97};
\draw[->] (1) to [out=45.0,in=90.0,looseness=5] (1);
\draw[->] (2) to [out=0.0,in=45.0,looseness=5] (2);
\draw[->] (3) -- (1);
\draw[->] (3) -- (2);
\draw[->] (3) to [out=-45.0,in=0.0,looseness=5] (3);
\draw[->] (4) to [out=-90.0,in=-45.0,looseness=5] (4);
\draw[->] (5) -- (2);
\draw[->] (5) -- (4);
\draw[->] (5) to [out=-135.0,in=-90.0,looseness=5] (5);
\draw[->] (6) to [out=-180.0,in=-135.0,looseness=5] (6);
\draw[->] (7) -- (2);
\draw[->] (7) -- (4);
\draw[->] (7) -- (6);
\draw[->] (7) to [out=-225.0,in=-180.0,looseness=5] (7);
\draw[->] (8) -- (2);
\draw[->] (8) -- (4);
\draw[->] (8) -- (5);
\draw[->] (8) -- (6);
\draw[->] (8) -- (7);
\draw[->] (8) to [out=-270.0,in=-225.0,looseness=5] (8);
\draw[->][densely dotted] (8) -- (1);
\draw[->][densely dotted] (8) -- (3);
\draw[->][densely dotted] (7) -- (1);
\draw[->][densely dotted] (7) -- (3);
\draw[->][densely dotted] (7) -- (5);
\draw[->][densely dotted] (6) -- (1);
\draw[->][densely dotted] (6) -- (2);
\draw[->][densely dotted] (6) -- (3);
\draw[->][densely dotted] (6) -- (4);
\draw[->][densely dotted] (6) -- (5);
\draw[->][densely dotted] (5) -- (1);
\draw[->][densely dotted] (5) -- (3);
\draw[->][densely dotted] (4) -- (1);
\draw[->][densely dotted] (4) -- (2);
\draw[->][densely dotted] (4) -- (3);
\draw[->][densely dotted] (2) -- (1);
}\\
А матрица линейного отношения $F_2$, дополненного до отношения линецйного порядка, будет соответствовать данной (все нули под главной диагональю становятся единицами):
$$ \left( \begin{array}{cccccccc}
		   1 &0 &0 &0 &0 &0 &0 &0 
        \\1 &1 &0 &0 &0 &0 &0 &0
        \\1 &1 &1 &0 &0 &0 &0 &0
        \\1 &1 &1 &1 &0 &0 &0 &0
        \\1 &1 &1 &1 &1 &0 &0 &0
        \\1 &1 &1 &1 &1 &1 &0 &0
        \\1 &1 &1 &1 &1 &1 &1 &0
        \\1 &1 &1 &1 &1 &1 &1 &1 \end{array} \right) $$
\end{proof}

\pagebreak
\begin{proof} \textbf{Бинарное отношение $F_3$}\\
Построим матрицу смежности для б.о. $F_3$
$$ \left( \begin{array}{cccccccc}
		   1 &0 &0 &0 &0 &0 &0 &0 
        \\0 &1 &0 &0 &0 &0 &0 &0
        \\0 &0 &1 &0 &0 &0 &0 &0
        \\0 &0 &0 &1 &0 &0 &0 &0
        \\0 &0 &0 &0 &1 &0 &0 &0
        \\0 &0 &0 &0 &0 &1 &0 &0
        \\0 &0 &0 &0 &0 &0 &1 &1
        \\0 &0 &0 &0 &0 &0 &1 &1 \end{array} \right) $$
Бинарное отношение $F_3$ рефлексивное, т.к все элементы матрицы смежности на главной диагонали равны единице, и симметричное, т.к. матрица зеркальна относительно главной диагонали.

\tikz {
	\path
(4.0, 8.0) node[state] (1) {19}
(6.83, 6.83) node[state] (2) {33}
(8.0, 4.0) node[state] (3) {69}
(6.83, 1.17) node[state] (4) {72}
(4.0, 0.0) node[state] (5) {77}
(1.17, 1.17) node[state] (6) {91}
(0.0, 4.0) node[state] (7) {96}
(1.17, 6.83) node[state] (8) {97};
\draw[->] (1) to [out=45.0,in=90.0,looseness=5] (1);
\draw[->] (2) to [out=0.0,in=45.0,looseness=5] (2);
\draw[->] (3) to [out=-45.0,in=0.0,looseness=5] (3);
\draw[->] (4) to [out=-90.0,in=-45.0,looseness=5] (4);
\draw[->] (5) to [out=-135.0,in=-90.0,looseness=5] (5);
\draw[->] (6) to [out=-180.0,in=-135.0,looseness=5] (6);
\draw[->] (7) to [out=-225.0,in=-180.0,looseness=5] (7);
\draw (7) -- (8);
\draw[->] (8) to [out=-270.0,in=-225.0,looseness=5] (8);
}\\
Бинарное отношение $F_3$ транзитивно, т.к. на графе отсутствуют пути длины 2.
\\ \\
Отношение $F_3$:
\begin{itemize}
    \item Рефлексивное
    \item Симметричное
    \item Транзитивное
\end{itemize}
Соответственно, оно является отношением эквивалентности.
Множество M разбивается на классы эквивалентности по целой части при делении на 5. А именно:
\begin{itemize}
    \item $\{19\}$ - целая часть = 2
    \item $\{33\}$ - целая часть = 6
    \item $\{69\}$ - целая часть = 13
    \item $\{72\}$ - целая часть = 14
    \item $\{77\}$ - целая часть = 15
    \item $\{91\}$ - целая часть = 18
    \item $\{96, 97\}$ - целая часть = 19
\end{itemize}
\end{proof}

% \pagebreak
\begin{proof} \textbf{Бинарное отношение $F_4$}\\
Построим матрицу смежности для б.о. $F_4$
$$ \left( \begin{array}{cccccccc}
		   0 &0 &0 &1 &0 &0 &1 &0 
        \\0 &0 &0 &1 &0 &0 &1 &0 
        \\0 &0 &0 &1 &0 &0 &1 &0 
        \\1 &1 &1 &0 &1 &1 &0 &1 
        \\0 &0 &0 &1 &0 &0 &1 &0 
        \\0 &0 &0 &1 &0 &0 &1 &0 
        \\1 &1 &1 &0 &1 &1 &0 &1 
        \\0 &0 &0 &1 &0 &0 &1 &0 \end{array} \right) $$
Бинарное отношение $F_4$ арефлексивное, т.к. элементы матрицы смежности на главной диагонали не равны единице, и  симметричное, т.к. матрица зеркальна относительно главной диагонали.


\tikz {
	\path
(4.0, 8.0) node[state] (1) {19}
(6.83, 6.83) node[state] (2) {33}
(8.0, 4.0) node[state] (3) {69}
(6.83, 1.17) node[state] (4) {72}
(4.0, 0.0) node[state] (5) {77}
(1.17, 1.17) node[state] (6) {91}
(0.0, 4.0) node[state] (7) {96}
(1.17, 6.83) node[state] (8) {97};
\draw (1) -- (4);
\draw (1) -- (7);
\draw (2) -- (4);
\draw (2) -- (7);
\draw (3) -- (4);
\draw (3) -- (7);
\draw (4) -- (5);
\draw (4) -- (6);
\draw (4) -- (8);
\draw (5) -- (7);
\draw (6) -- (7);
\draw (7) -- (8);
}\\
Бинарное отношение $F_4$ не транзитивно. Например, между вершинами 96 и 72 есть несколько путей длины 2, но нет ни одного пути длиной 1.
\\ \\
Отношение $F_4$:
\begin{itemize}
    \item Арефлексивное
    \item Симметричное
    \item Не транзитивное
\end{itemize}

Для построения транзитивного замыкания применим алгоритм Уоршелла:
$$ \left( \begin{array}{cccccccc}
		   0 &0 &0 &1 &0 &0 &1 &0 
        \\0 &0 &0 &1 &0 &0 &1 &0 
        \\0 &0 &0 &1 &0 &0 &1 &0 
        \\1 &1 &1 &0 &1 &1 &0 &1 
        \\0 &0 &0 &1 &0 &0 &1 &0 
        \\0 &0 &0 &1 &0 &0 &1 &0 
        \\1 &1 &1 &0 &1 &1 &0 &1 
        \\0 &0 &0 &1 &0 &0 &1 &0 \end{array} \right) \Rightarrow
    \left( \begin{array}{cccccccc}
		   0 &0 &0 &1 &0 &0 &1 &0 
        \\0 &0 &0 &1 &0 &0 &1 &0 
        \\0 &0 &0 &1 &0 &0 &1 &0 
        \\1 &1 &1 &\textcolor{red}{1} &1 &1 &\textcolor{red}{1} &1 
        \\0 &0 &0 &1 &0 &0 &1 &0 
        \\0 &0 &0 &1 &0 &0 &1 &0 
        \\1 &1 &1 &\textcolor{red}{1} &1 &1 &\textcolor{red}{1} &1 
        \\0 &0 &0 &1 &0 &0 &1 &0 \end{array} \right) \Rightarrow
    \left( \begin{array}{cccccccc}
		   \textcolor{red}{1} &\textcolor{red}{1} &\textcolor{red}{1} &1 &\textcolor{red}{1} &\textcolor{red}{1} &1 &\textcolor{red}{1} 
        \\\textcolor{red}{1} &\textcolor{red}{1} &\textcolor{red}{1} &1 &\textcolor{red}{1} &\textcolor{red}{1} &1 &\textcolor{red}{1} 
        \\\textcolor{red}{1} &\textcolor{red}{1} &\textcolor{red}{1} &1 &\textcolor{red}{1} &\textcolor{red}{1} &1 &\textcolor{red}{1} 
        \\1 &1 &1 &1 &1 &1 &1 &1 
        \\\textcolor{red}{1} &\textcolor{red}{1} &\textcolor{red}{1} &1 &\textcolor{red}{1} &\textcolor{red}{1} &1 &\textcolor{red}{1} 
        \\\textcolor{red}{1} &\textcolor{red}{1} &\textcolor{red}{1} &1 &\textcolor{red}{1} &\textcolor{red}{1} &1 &\textcolor{red}{1} 
        \\1 &1 &1 &1 &1 &1 &1 &1 
        \\\textcolor{red}{1} &\textcolor{red}{1} &\textcolor{red}{1} &1 &\textcolor{red}{1} &\textcolor{red}{1} &1 &\textcolor{red}{1} \end{array} \right)    
        $$
\end{proof}

\pagebreak
\begin{proof} \textbf{Бинарное отношение $F_5$}\\
Построим матрицу смежности для б.о. $F_5$\\
$$ \left( \begin{array}{cccccccc}
          1 &0 &0 &0 &0 &0 &0 &0
        \\0 &1 &0 &0 &0 &0 &0 &0
        \\0 &0 &1 &1 &1 &0 &0 &0
        \\0 &0 &1 &1 &1 &0 &0 &0
        \\0 &0 &1 &1 &1 &0 &0 &0
        \\0 &0 &0 &0 &0 &1 &1 &1
        \\0 &0 &0 &0 &0 &1 &1 &1
        \\0 &0 &0 &0 &0 &1 &1 &1 \end{array} \right) $$
Бинарное отношение $F_5$ рефлексивное, т.к все элементы матрицы смежности на главной диагонали равны единице, и симметричное, т.к. матрица зеркальна относительно главной диагонали.


\tikz {
	\path
(4.0, 8.0) node[state] (1) {19}
(6.83, 6.83) node[state] (2) {33}
(8.0, 4.0) node[state] (3) {69}
(6.83, 1.17) node[state] (4) {72}
(4.0, 0.0) node[state] (5) {77}
(1.17, 1.17) node[state] (6) {91}
(0.0, 4.0) node[state] (7) {96}
(1.17, 6.83) node[state] (8) {97};
\draw[->] (1) to [out=45.0,in=90.0,looseness=5] (1);
\draw[->] (2) to [out=0.0,in=45.0,looseness=5] (2);
\draw[->] (3) to [out=-45.0,in=0.0,looseness=5] (3);
\draw (3) -- (4);
\draw (3) -- (5);
\draw[->] (4) to [out=-90.0,in=-45.0,looseness=5] (4);
\draw (4) -- (5);
\draw[->] (5) to [out=-135.0,in=-90.0,looseness=5] (5);
\draw[->] (6) to [out=-180.0,in=-135.0,looseness=5] (6);
\draw (6) -- (7);
\draw (6) -- (8);
\draw[->] (7) to [out=-225.0,in=-180.0,looseness=5] (7);
\draw (7) -- (8);
\draw[->] (8) to [out=-270.0,in=-225.0,looseness=5] (8);
}\\
Бинарное отношение $F_5$ транзитивно, т.к. для всех вершин, между которыми есть путь длины 2, есть и путь длины 1.
\\ \\
Отношение $F_5$:
\begin{itemize}
    \item Рефлексивное
    \item Симметричное
    \item Транзитивное
\end{itemize}
Соответственно оно является отношением эквивалентности.
\\ \\
Множество M будет разбито на классы эквивалентности по разности между элементами < 10, а именно:
\begin{itemize}
    \item $\{19\}$ - нет больше элементов, разность с которыми будет < 10
    \item $\{33\}$ - нет больше элементов, разность с которыми будет < 10
    \item $\{69, 72, 77\}$ - разность между любыми 2-мя элементами < 10
    \item $\{91, 96, 97\}$ - разность между любыми 2-мя элементами < 10
\end{itemize}
\end{proof}





        
  % для удобства создаём по аналогии файлы ihw1.tex, ihw2.tex, etc
	                  % и просто меняем имя при компиляции
\end{document}