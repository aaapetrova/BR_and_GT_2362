\documentclass{amsart}

\usepackage[utf8]{inputenc}
\usepackage[T2A]{fontenc}
\usepackage[english,russian]{babel}
\usepackage{amsthm,amsmath,amsfonts,amssymb}
\usepackage{fullpage}
\usepackage{eufrak}

%%% Дополнительная работа с математикой
\usepackage{amsfonts,amssymb,amsthm,mathtools} % AMS
\usepackage{amsmath}
\usepackage{icomma}

%% Шрифты
\usepackage{euscript}	 % Шрифт Евклид
\usepackage{mathrsfs} % Красивый матшрифт

%% Свои команды
\DeclareMathOperator{\sgn}{\mathop{sgn}}	% сигнум
\DeclareMathOperator{\cov}{\mathop{cov}}	% ковариация
\DeclareMathOperator{\lb}{\mathop{lb}}	% бинарный логарифм (логарифм по основанию 2)
\DeclareMathOperator{\supp}{\mathop{supp}}	% носитель

\renewcommand{\Im}{\mathop{\mathrm{Im}}\nolimits}	% мнимая часть
\renewcommand{\Re}{\mathop{\mathrm{Re}}\nolimits}	% вещественная часть

\renewcommand{\Prob}{\mathbb P}	% вероятность
\newcommand{\Expect}{\mathbb E}	% математическое ожидание
\renewcommand{\Variance}{\mathbb D}	% дисперсия
\newcommand{\Entropy}{\mathbb H}	% энтропия

%% Перенос знаков в формулах (по Львовскому)
\newcommand*{\hm}[1]{#1\nobreak\discretionary{}
	{\hbox{$\mathsurround=0pt #1$}}{}}

%%% Работа с картинками
\usepackage{graphicx}  % Для вставки рисунков
\graphicspath{{images/}{images2/}}  % папки с картинками
\setlength\fboxsep{3pt} % Отступ рамки \fbox{} от рисунка
\setlength\fboxrule{1pt} % Толщина линий рамки \fbox{}
\usepackage{wrapfig} % Обтекание рисунков и таблиц текстом
\RequirePackage{caption}
\DeclareCaptionLabelSeparator{defffis}{ -- }
\captionsetup{justification=centering,labelsep=defffis}

\renewcommand{\qedsymbol}{}

%%% Работа с таблицами
\usepackage{array,tabularx,tabulary,booktabs} % Дополнительная работа с таблицами
\usepackage{longtable}  % Длинные таблицы
\usepackage{multirow} % Слияние строк в таблице

\newtheorem{problem}{Задание}

\begin{document}
	
	\newcommand{\problemset}[1]{
		\begin{center}
			\Large #1
		\end{center}
	}
	
	\begin{tabbing}
	\hspace{11cm} \= Студент: \= Коротков Фёдор \\ % не забудьте исправить, студент Вы или студентка :)
																									% (а то некоторые забывают)
	\> Группа: \> 2362 \\  % Здесь меняете № группы
	\> Вариант: \> QG \\    % А здесь меняете № варианта
	\> Дата: \> \today     % А вот здесь ничего не меняем!!!
\end{tabbing}
\hrule
\vspace{1cm}  % в данном файле меняем только Пол, Фамилию Имя, № группы и № варианта
	\problemset{Комбинаторика и теория графов}
\problemset{Индивидуальное домашнее задание №1}	% поменяйте номер ИДЗ

\renewcommand*{\proofname}{Решение}

Дано множество \emph{M} = \{34, 71, 42, 45, 52, 89, 28, 29\}

\begin{problem}
\[ F_1(x,y) = 1 \Leftrightarrow \exists z \in M : (x-z)(y-z)<0; \]
\end{problem}

\begin{proof} 

\end{proof}

\begin{problem}
\[ F_2(x,y) = 1 \Leftrightarrow x \geqslant	 y \emph{ порязрядно}; \]
\end{problem}

\begin{proof}

\end{proof}

\begin{problem}
\[ F_3(x,y) = 1 \Leftrightarrow \left[ \frac{x}{5} \right] = \left[ \frac{y}{5} \right]; \]
\end{problem}

\begin{proof}

\end{proof}

\begin{problem}
\[ F_4(x,y) = 1 \Leftrightarrow x^2-y^3	\emph{ четно}; \]
\end{problem}

\begin{proof}

\end{proof}

\begin{problem}
\[ F_5(x,y) = 1 \Leftrightarrow |x-y|<10. \]
\end{problem}

\begin{proof}

\end{proof}  % для удобства создаём по аналогии файлы ihw1.tex, ihw2.tex, etc
	                  % и просто меняем имя при компиляции
\end{document}