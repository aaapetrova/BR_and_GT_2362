%% Решение задания №1
\pagebreak
\begin{proof} $ $\\
    \begin{itemize}
        \item Не является эейлеровым, т.к. не все вершины имеют четную степень
        \item Является полуэйлеровым, т.к. ровно 2 вершины имеют нечентную степень: B, J
        \item Не является гамильтоновым
        \item Является полугамильтоновым, т.к. существует путь, проходящий по всем вершинам ровно 1 раз: B, A, D, E, F, J, I, K, L, H, G, C
        \item Не является двудольным, т.к. смежные вершины I, J, F оказываются одного цвета при раскрашивании графа
        \begin{center}
        \begin{tikzcd}
		& \textcolor{red}{A}  \ar[r, dash]  \ar[d, dash] 
		& \textcolor{green}{B}  
		\\
		\textcolor{green}{C}   \ar[dr, dash] \ar[d, dash]  
		& \textcolor{green}{D} \ar[dl, dash] \ar[r, dash] \ar[d, dash]   
		& \textcolor{red}{E} \ar[d, dash] \ar[r, dash]  
		& \textcolor{green}{F} \ar[d, dash] 
		\\
		\textcolor{red}{G}   \ar[ur, dash]   
		& \textcolor{red}{H} \ar[u, dash] 
		& \textcolor{green}{I} \ar[l, dash] \ar[r, dash]
		& \textcolor{green}{J} \ar[ul, dash]
        \\
        & \textcolor{red}{K} \ar[ur, dash] \ar[r, dash]
        & \textcolor{green}{L} \ar[ul, dash]
	    \end{tikzcd}
        \end{center}
        \item Не является вершинно двусвязным, т.к. в графе присутствуют шарниры: A, D
        \item Не является реберно двусвязным, т.к. в графе присутствуют мосты: BA, AD
        \item Дерево блоков и точек сочленения:\\
        \tikz {
        \path
        (0, 0) node[white, circle,fill=black] (AB) {AB}
        (2, 0) node[circle,fill=red] (A) {A}
        (4, 0) node[white, circle,fill=black] (AD) {AD}
        (6, 0) node[circle,fill=red] (D) {D}
        (8, 0) node[white, circle, fill=black] (other) {DGCHLKIJFE};

        \draw (AB) -- (A);
        \draw (A) -- (AD);
        \draw (AD) -- (D);
        \draw (D) -- (other);
        }

    \end{itemize}
\end{proof}