\problemset{Комбинаторика и теория графов}
\problemset{Индивидуальное домашнее задание №2}	% поменяйте номер ИДЗ

\renewcommand*{\proofname}{Решение}
%%%%%%%%%%%%%% ЗАДАНИЕ №1 %%%%%%%%%%%%%%
%% Условие задания №1
\begin{problem}
	Определить, является ли данный граф эйлеровым, полуэйлеровым, гамильтоновым, полугамильтоновым, двудольным, вершинно-двусвязным, рёберно-двусвянзным. Построить дерево блоков и точек сочленения.
    \begin{center}
    \begin{tikzcd}
		& A  \ar[r, dash]  \ar[d, dash] 
		& B  
		\\
		C   \ar[dr, dash] \ar[d, dash]  
		& D \ar[dl, dash] \ar[r, dash] \ar[d, dash]   
		& E \ar[d, dash] \ar[r, dash]  
		& F \ar[d, dash] 
		\\
		G   \ar[ur, dash]   
		& H \ar[u, dash] 
		& I \ar[l, dash] \ar[r, dash]
		& J \ar[ul, dash]
        \\
        & K \ar[ur, dash] \ar[r, dash]
        & L \ar[ul, dash]
	\end{tikzcd}\\
    \end{center}
\end{problem}

%% Решение 1
%% Решение задания №1
\pagebreak
\begin{proof} $ $\\
    \begin{itemize}
        \item Не является эейлеровым, т.к. не все вершины имеют четную степень
        \item Является полуэйлеровым, т.к. ровно 2 вершины имеют нечентную степень: B, J
        \item Не является гамильтоновым
        \item Является полугамильтоновым, т.к. существует путь, проходящий по всем вершинам ровно 1 раз: B, A, D, E, F, J, I, K, L, H, G, C
        \item Не является двудольным, т.к. смежные вершины I, J, F оказываются одного цвета при раскрашивании графа
        \begin{center}
        \begin{tikzcd}
		& \textcolor{red}{A}  \ar[r, dash]  \ar[d, dash] 
		& \textcolor{green}{B}  
		\\
		\textcolor{green}{C}   \ar[dr, dash] \ar[d, dash]  
		& \textcolor{green}{D} \ar[dl, dash] \ar[r, dash] \ar[d, dash]   
		& \textcolor{red}{E} \ar[d, dash] \ar[r, dash]  
		& \textcolor{green}{F} \ar[d, dash] 
		\\
		\textcolor{red}{G}   \ar[ur, dash]   
		& \textcolor{red}{H} \ar[u, dash] 
		& \textcolor{green}{I} \ar[l, dash] \ar[r, dash]
		& \textcolor{green}{J} \ar[ul, dash]
        \\
        & \textcolor{red}{K} \ar[ur, dash] \ar[r, dash]
        & \textcolor{green}{L} \ar[ul, dash]
	    \end{tikzcd}
        \end{center}
        \item Не является вершинно двусвязным, т.к. в графе присутствуют шарниры: A, D
        \item Не является реберно двусвязным, т.к. в графе присутствуют мосты: BA, AD
        \item Дерево блоков и точек сочленения:\\
        \tikz {
        \path
        (0, 0) node[white, circle,fill=black] (AB) {AB}
        (2, 0) node[circle,fill=red] (A) {A}
        (4, 0) node[white, circle,fill=black] (AD) {AD}
        (6, 0) node[circle,fill=red] (D) {D}
        (8, 0) node[white, circle, fill=black] (other) {DGCHLKIJFE};

        \draw (AB) -- (A);
        \draw (A) -- (AD);
        \draw (AD) -- (D);
        \draw (D) -- (other);
        }

    \end{itemize}
\end{proof}

%% Условие задания 2
\begin{problem}
Найдите хроматический многочлен данного графа:
\begin{center}
    \begin{tikzcd}
		& A  \ar[r, dash]  \ar[d, dash] 
		& B  \ar[dl, dash]
		\\
		C   \ar[r, dash] \ar[dr, dash]  
		& D \ar[r, dash] \ar[d, dash]   
		& E \ar[d, dash] \ar[r, dash] \ar[dr, dash]
		& F
		\\
		G   \ar[r, dash]   
		& H \ar[r, dash] 
		& I \ar[d, dash]
		& J \ar[ul, dash]
        \\
        & K \ar[r, dash]
        & L
	\end{tikzcd}\\
    \end{center}
\end{problem}
\begin{proof} $ $\\
    Обозначим исодный граф за Gr\\
    $P_G = P_J = P_F = (t-1)$\\
    Удалим вершины G, J, F из графа:
    \begin{center}
    \begin{tikzcd}
		& A  \ar[r, dash]  \ar[d, dash] 
		& B  \ar[dl, dash]
		\\
		C   \ar[r, dash] \ar[dr, dash]  
		& D \ar[r, dash] \ar[d, dash]   
		& E \ar[d, dash]
		\\
		& H \ar[r, dash] 
		& I \ar[d, dash]
        \\
        & K \ar[r, dash]
        & L
	\end{tikzcd}\\
    \end{center}
    $P_{KL} = (t-1)^2$\\
    Удалим вершины K, L из графа:
    \begin{center}
    \begin{tikzcd}
		& A  \ar[r, dash]  \ar[d, dash] 
		& B  \ar[dl, dash]
		\\
		C   \ar[r, dash] \ar[dr, dash]  
		& D \ar[r, dash] \ar[d, dash]   
		& E \ar[d, dash]
		\\
		& H \ar[r, dash] 
		& I
	\end{tikzcd}\\
    \end{center}
    $P_C = (t-2)$\\
    Удалим вершину C из графа:
    \begin{center}
    \begin{tikzcd}
		& A  \ar[r, dash]  \ar[d, dash] 
		& B  \ar[dl, dash]
		\\
		& D \ar[r, dash] \ar[d, dash]   
		& E \ar[d, dash]
		\\
		& H \ar[r, dash] 
		& I
	\end{tikzcd}\\
    \end{center}
    $P_{AB} = (t-1)(t-2)$\\
    Удалим вершины A, B графа:
    \begin{center}
    \begin{tikzcd}
		& D \ar[r, dash] \ar[d, dash]   
		& E \ar[d, dash]
		\\
		& H \ar[r, dash] 
		& I
	\end{tikzcd}\\
    \end{center}
    $P_{DEFI} = P_{C_4} = (t-1)^4+(t-1)$\\
    $P_{Gr} = P_{DEFI} \cdot P_{AB} \cdot P_{C} \cdot P_{KL} \cdot P_G \cdot P_J \cdot P_F = [(t-1)^4+(t-1)](t-1)^6(t-2)^2 = (t-1)^7(t-2)^2[(t-1)^3 + 1]$\\\\
    \textbf{Ответ: } $P_{Gr} = (t-1)^7(t-2)^2[(t-1)^3 + 1]$
    
\end{proof}

%% Условие задания 3
\begin{problem}
Из полного графа на 133 вершинах, удалили рёбра AB,
BC, EF и DF. Постройте хроматический многочлен получив-
шегося графа. Упрощать ответ не обязательно.
\end{problem}
\newcommand{\ols}[1]{\mskip.5\thinmuskip\overline{\mskip-.5\thinmuskip {#1} \mskip-.5\thinmuskip}\mskip.5\thinmuskip} % overline short

\begin{proof} $ $\\
    Обозначим исходный граф за G.\\\\
    Хроматический многочлен графа с объединенными вершинами будем обозначать как $P_{XY}$, где X, Y - вершины, которые были объединены\\\\
    Хроматический многочлен графа с добавленным ребром между вершинами будем обозначать как $P_{\ols{XY}}$, где X, Y - вершины, вершины, между которыми было проведено ребро.\\\\
    Изобразим ребра, которые были удалены из графа:
    \begin{center}
    \begin{tikzcd}
    &B \ar[d, dash, dashed]
    &E \ar[d, dash, dashed]
    &D \ar[dl, dash, dashed]
    \\
    &C
    &F
    \end{tikzcd}
    \end{center}
    $P_G = P_{BC} + P_{\ols{BC}}$\\
    \begin{tikzcd}
    \\
    &&&D \ar[dd, dash, dashed]
    \\
    &BC \ar[r, dash] \ar[urr, dash] \ar[drr, dash]
    &E \ar[ur, dash] \ar[dr, dash, dashed]
    \\
    &&&F
    \end{tikzcd}
    \begin{tikzcd}
    &B \ar[d, dash]
    &E \ar[d, dash, dashed]
    &D \ar[dl, dash, dashed]
    \\
    &C
    &F
    \end{tikzcd}\\
    $P_{BC} = P_{EF} + P_{\ols{EF}}$\\
    $P_{EF} = P_{K_{131}}$\\
    $P_{\ols{EF}} = P_{\ols{FD}} + P_{FD} = P_{K_{132}} + P_{K_{131}} \Rightarrow P_{BC} = 2P_{K_{131}}+P_{K_{132}}$\\\\
    $P_{\ols{BC}} = P_{FE} + P_{\ols{FE}}$\\
    $P_{FE} = P_{K_{132}}$\\
    $P_{\ols{FE}} = P_{DF} + P_{\ols{DF}} = P_{K_{132}} + P_{K_{133}} \Rightarrow 2P_{K_{132}}+P_{K_{133}}$\\\\
    $P_G = 3P_{K_{132}} + 2P_{K_{131}} + P_{K_{133}}$\\\\
    \textbf{Ответ:} $3A_t^{132} + 2A_t^{131} + A_t^{133}$
\end{proof}

\vspace{10mm}
%% Условие задания №4
\begin{problem} $ $\\
	\begin{itemize}
	    \item[a)] построить код Прюфера для данного дерева:
        \vspace{4mm}
        \begin{center}
        \begin{tikzcd}
        1 \ar[d, dash] \ar[r, dash]
        &2
        &3 \ar[dll, dash]
        &4 \ar[r, dash]
        &5 \ar[r, dash] \ar[dllll, dash] \ar[dlll, dash]
        &6 \ar[dl, dash]
        \\
        7 
        &8 \ar[r, dash]
        &9 \ar[r, dash]
        &10
        &11
        \end{tikzcd}
        \end{center}
        \vspace{4mm}
        \item [б)] Построить дерево по коду Прюфера: 2 11 2 3 3 4 5 9 5
	\end{itemize}
\end{problem}

%% Решение задания 4
\begin{proof} $ $\\
    \begin{itemize}
        \item[a)] Код Прюфера: 1 7 7 5 5 9 8 5 6
        \item[б)] Получившиеся дерево по коду Прюфера:
        \begin{center}
        \begin{tikzcd}
        1 \ar[r, dash]
        &2 \ar[dl, dash] \ar[r, dash]
        &3 \ar[dl, dash] \ar[r, dash]
        &4 \ar[r, dash]
        &5 \ar[d, dash] \ar[dll, dash]
        &6 \ar[dl, dash]
        \\
        7 
        &8 
        &9 \ar[r, dash]
        &10
        &11
        \end{tikzcd}
        \end{center}
    \end{itemize}
\end{proof}

%% Условие задания №5
\begin{problem} При помощи плгоритма Kosaraju найти компоненты сильной связности данного графа:\\
    \begin{center}
        \begin{tikzcd}
            A \ar[r, rightarrow]
            &B \ar[d, rightarrow] \ar[ddl, leftrightarrow]
            &C \ar[r, leftrightarrow]
            &D \ar[llldd, rightarrow]
            \\
            E 
            &F \ar[urr, rightarrow]
            &G \ar[ur, rightarrow] \ar[r, rightarrow] \ar[ddr, leftrightarrow]
            &H 
            \\
            I \ar[r, leftarrow]
            &J \ar[ld, leftrightarrow]
            &K \ar[dl,  rightarrow]
            &L \ar[d, leftrightarrow]
            \\ 
            M \ar[u, rightarrow]
            &N \ar[l, leftrightarrow]
            &O \ar[u, rightarrow]
            &P \ar[l, rightarrow]
        \end{tikzcd}
    \end{center}
\end{problem}

%% Решение задания 5
\begin{proof} $ $\\
    Начинаем поиски в глубину с вершин: A, E, L\\
    Полученный стек: [I, C, D, F, B, A, E, H, G, J, M, N, K, O, P, L\\
    Транспонированный граф:
        \begin{center}
        \begin{tikzcd}
            A \ar[r, leftarrow]
            &B \ar[d, leftarrow] \ar[ddl, leftrightarrow]
            &C \ar[r, leftrightarrow]
            &D \ar[llldd, leftarrow]
            \\
            E 
            &F \ar[urr, leftarrow]
            &G \ar[ur, leftarrow] \ar[r, leftarrow] \ar[ddr, leftrightarrow]
            &H 
            \\
            I \ar[r, rightarrow]
            &J \ar[ld, leftrightarrow]
            &K \ar[dl,  leftarrow]
            &L \ar[d, leftrightarrow]
            \\ 
            M \ar[u, leftarrow]
            &N \ar[l, leftrightarrow]
            &O \ar[u, leftarrow]
            &P \ar[l, leftarrow]
        \end{tikzcd}
        \end{center}
        \vspace{5mm}
    Граф после поисков в глубину в порядке доставания вершин из стека и окрашивания вершин в рамках одного поиска:
        \begin{center}
        \begin{tikzcd}
            \textcolor{pink}{A} \ar[r, leftarrow]
            &\textcolor{blue}{B} \ar[d, leftarrow] \ar[ddl, leftrightarrow]
            &\textcolor{blue}{C} \ar[r, leftrightarrow]
            &\textcolor{blue}{D} \ar[llldd, leftarrow]
            \\
            \textcolor{orange}{E} 
            &\textcolor{blue}{F} \ar[urr, leftarrow]
            &\textcolor{red}{G} \ar[ur, leftarrow] \ar[r, leftarrow] \ar[ddr, leftrightarrow]
            &\textcolor{brown}{H} 
            \\
            \textcolor{blue}{I} \ar[r, rightarrow]
            &\textcolor{green}{J} \ar[ld, leftrightarrow]
            &K \ar[dl,  leftarrow]
            &\textcolor{red}{L} \ar[d, leftrightarrow]
            \\ 
            \textcolor{green}{M} \ar[u, leftarrow]
            &\textcolor{green}{N} \ar[l, leftrightarrow]
            &\textcolor{yellow}{O} \ar[u, leftarrow]
            &\textcolor{red}{P} \ar[l, leftarrow]
        \end{tikzcd}
        \end{center}
        Таким образом, граф герца для данного графа:
        \begin{center}
        \begin{tikzcd}
                    E
        &LPG \ar[dl, rightarrow] \ar[r, rightarrow] \ar[drr, rightarrow]
        &H
        &
        \\
        O \ar[r, rightarrow]
        &K \ar[r, rightarrow]
        &NMJ \ar[r, rightarrow]
        &BFIDC
        \\
        &
        &A \ar[ur, rightarrow]
        \end{tikzcd}
        \end{center}
\end{proof}

%% Условие задания №6
\begin{problem}
Найдите максимальный поток через данную плоскую
сеть:
\begin{center}
    \usetikzlibrary{graphs,automata,positioning}
    \tikz {
    	\path
    (2, 0) node (B) {B}
    (4, 0) node (E) {E}
    (6, 0) node (H) {H}
    (8, 0) node (K) {K}
    (0, -2) node (A) {A}
    (2, -2) node (C) {C}
    (4, -2) node (F) {F}
    (6, -2) node (I) {I}
    (8, -2) node (L) {L}
    (10, -2) node (N) {N}
    (2, -4) node (D) {D}
    (4, -4) node (G) {G}
    (6, -4) node (J) {J}
    (8, -4) node (M) {M};
    \path[->]
    (A) edge node[sloped,above] {22} (B)
    (A) edge node[sloped,above] {9} (C)
    (A) edge node[sloped,above] {19} (D)
    (B) edge node[sloped,above] {2} (E)
    (B) edge node[sloped,above] {3} (F)
    (C) edge node[sloped,above] {1} (F)
    (C) edge node[sloped,above] {9} (G)
    (D) edge node[sloped,above] {8} (G)
    (E) edge node[sloped,above] {5} (H)
    (F) edge node[sloped,above] {3} (H)
    (F) edge node[sloped,above] {10} (I)
    (F) edge node[sloped,above] {8} (J)
    (G) edge node[sloped,above] {6} (J)
    (H) edge node[sloped,above] {9} (K)
    (H) edge node[sloped,above] {5} (L)
    (I) edge node[sloped,above] {7} (L)
    (J) edge node[sloped,above] {4} (M)
    (J) edge node[sloped,above] {4} (L)
    (K) edge node[sloped,above] {14} (N)
    (L) edge node[sloped,above] {12} (N)
    (M) edge node[sloped,above] {10} (N);
    }
\end{center}
\end{problem}
%% Решение задания 6
\begin{proof} $ $\\
ABEHK - 2\\
Остаточная сеть:\\
    \usetikzlibrary{graphs,automata,positioning}
    \tikz {
    	\path
    (2, 0) node (B) {B}
    (4, 0) node (E) {E}
    (6, 0) node (H) {H}
    (8, 0) node (K) {K}
    (0, -2) node (A) {A}
    (2, -2) node (C) {C}
    (4, -2) node (F) {F}
    (6, -2) node (I) {I}
    (8, -2) node (L) {L}
    (10, -2) node (N) {N}
    (2, -4) node (D) {D}
    (4, -4) node (G) {G}
    (6, -4) node (J) {J}
    (8, -4) node (M) {M};
    \path[->]
    (A) edge[ultra thick] node[sloped,above] {20} (B)
    (A) edge node[sloped,above] {9} (C)
    (A) edge node[sloped,above] {19} (D)
    (A) edge[<-, bend right =-40] node[sloped,above] {2} (B)
    (B) edge[ultra thick] node[sloped, above] {0} (E)
    (B) edge[<-, bend right =-40] node[sloped,above] {2} (E)
    (B) edge node[sloped,above] {3} (F)
    (C) edge node[sloped,above] {1} (F)
    (C) edge node[sloped,above] {9} (G)
    (D) edge node[sloped,above] {8} (G)
    (E) edge[ultra thick] node[sloped,above] {3} (H)
    (E) edge[<-, bend right =-40] node[sloped,above] {2} (H)
    (F) edge node[sloped,above] {3} (H)
    (F) edge node[sloped,above] {10} (I)
    (F) edge node[sloped,above] {8} (J)
    (G) edge node[sloped,above] {6} (J)
    (H) edge[ultra thick] node[sloped,above] {7} (K)
    (H) edge[<-, bend right =-40] node[sloped,above] {2} (K)
    (H) edge node[sloped,above] {5} (L)
    (I) edge node[sloped,above] {7} (L)
    (J) edge node[sloped,above] {4} (M)
    (J) edge node[sloped,above] {4} (L)
    (K) edge[ultra thick] node[sloped,above] {12} (N)
    (K) edge[<-, bend right =-40] node[sloped,above] {2} (N)
    (L) edge node[sloped,above] {12} (N)
    (M) edge node[sloped,above] {10} (N);
    }\\\\
    ABFHKN - 3\\
    Остаточная сеть:\\
    \usetikzlibrary{graphs,automata,positioning}
    \tikz {
    	\path
    (2, 0) node (B) {B}
    (4, 0) node (E) {E}
    (6, 0) node (H) {H}
    (8, 0) node (K) {K}
    (0, -2) node (A) {A}
    (2, -2) node (C) {C}
    (4, -2) node (F) {F}
    (6, -2) node (I) {I}
    (8, -2) node (L) {L}
    (10, -2) node (N) {N}
    (2, -4) node (D) {D}
    (4, -4) node (G) {G}
    (6, -4) node (J) {J}
    (8, -4) node (M) {M};
    \path[->]
    (A) edge[ultra thick] node[sloped,above] {17} (B)
    (A) edge[<-, bend right =-40] node[sloped,above] {5} (B)
    (A) edge node[sloped,above] {9} (C)
    (A) edge node[sloped,above] {19} (D)
    (B) edge[<-, bend right =-40] node[sloped,above] {2} (E)
    (B) edge[ultra thick] node[sloped,above] {0} (F)
    (B) edge[<-, bend right =-30] node[sloped,above] {3} (F)
    (C) edge node[sloped,above] {1} (F)
    (C) edge node[sloped,above] {9} (G)
    (D) edge node[sloped,above] {8} (G)
    (E) edge node[sloped,above] {3} (H)
    (E) edge[<-, bend right =-40] node[sloped,above] {2} (H)
    (F) edge[ultra thick] node[sloped,above] {0} (H)
    (F) edge[<-, bend right =-30] node[sloped,above] {3} (H)
    (F) edge node[sloped,above] {10} (I)
    (F) edge node[sloped,above] {8} (J)
    (G) edge node[sloped,above] {6} (J)
    (H) edge[ultra thick] node[sloped,above] {4} (K)
    (H) edge[<-, bend right =-40] node[sloped,above] {5} (K)
    (H) edge node[sloped,above] {5} (L)
    (I) edge node[sloped,above] {7} (L)
    (J) edge node[sloped,above] {4} (M)
    (J) edge node[sloped,above] {4} (L)
    (K) edge[ultra thick] node[sloped,above] {9} (N)
    (K) edge[<-, bend right =-40] node[sloped,above] {5} (N)
    (L) edge node[sloped,above] {12} (N)
    (M) edge node[sloped,above] {10} (N);
    }\\\\
    ACFILN - 1\\
    Остаточная сеть:\\
    \usetikzlibrary{graphs,automata,positioning}
    \tikz {
    	\path
    (2, 0) node (B) {B}
    (4, 0) node (E) {E}
    (6, 0) node (H) {H}
    (8, 0) node (K) {K}
    (0, -2) node (A) {A}
    (2, -2) node (C) {C}
    (4, -2) node (F) {F}
    (6, -2) node (I) {I}
    (8, -2) node (L) {L}
    (10, -2) node (N) {N}
    (2, -4) node (D) {D}
    (4, -4) node (G) {G}
    (6, -4) node (J) {J}
    (8, -4) node (M) {M};
    \path[->]
    (A) edge node[sloped,above] {17} (B)
    (A) edge[<-, bend right =-40] node[sloped,above] {5} (B)
    (A) edge[ultra thick] node[sloped,above] {8} (C)
    (A) edge[<-, bend right =-40] node[sloped,above] {1} (C)
    (A) edge node[sloped,above] {19} (D)
    (B) edge[<-, bend right =-40] node[sloped,above] {2} (E)
    (B) edge[<-, bend right =-30] node[sloped,above] {3} (F)
    (C) edge[ultra thick] node[sloped,above] {0} (F)
    (C) edge[<-, bend right =-40] node[sloped,above] {1} (F)
    (C) edge node[sloped,above] {9} (G)
    (D) edge node[sloped,above] {8} (G)
    (E) edge node[sloped,above] {3} (H)
    (E) edge[<-, bend right =-40] node[sloped,above] {2} (H)
    (F) edge[<-, bend right =-30] node[sloped,above] {3} (H)
    (F) edge[ultra thick] node[sloped,above] {9} (I)
    (F) edge[<-, bend right =-40] node[sloped,above] {1} (I)
    (F) edge node[sloped,above] {8} (J)
    (G) edge node[sloped,above] {6} (J)
    (H) edge node[sloped,above] {4} (K)
    (H) edge[<-, bend right =-40] node[sloped,above] {5} (K)
    (H) edge node[sloped,above] {5} (L)
    (I) edge[ultra thick] node[sloped,above] {6} (L)
    (I) edge[<-, bend right =-40] node[sloped,above] {1} (L)
    (J) edge node[sloped,above] {4} (M)
    (J) edge node[sloped,above] {4} (L)
    (K) edge node[sloped,above] {9} (N)
    (K) edge[<-, bend right =-40] node[sloped,above] {5} (N)
    (L) edge[ultra thick] node[sloped,above] {11} (N)
    (L) edge[<-, bend right =-40] node[sloped,above] {1} (N)
    (M) edge node[sloped,above] {10} (N);
    }\\\\
    ACGJLN - 4\\
    Остаточная сеть:\\
    \usetikzlibrary{graphs,automata,positioning}
    \tikz {
    	\path
    (2, 0) node (B) {B}
    (4, 0) node (E) {E}
    (6, 0) node (H) {H}
    (8, 0) node (K) {K}
    (0, -2) node (A) {A}
    (2, -2) node (C) {C}
    (4, -2) node (F) {F}
    (6, -2) node (I) {I}
    (8, -2) node (L) {L}
    (10, -2) node (N) {N}
    (2, -4) node (D) {D}
    (4, -4) node (G) {G}
    (6, -4) node (J) {J}
    (8, -4) node (M) {M};
    \path[->]
    (A) edge node[sloped,above] {17} (B)
    (A) edge[<-, bend right =-40] node[sloped,above] {5} (B)
    (A) edge[ultra thick] node[sloped,above] {4} (C)
    (A) edge[<-, bend right =-40] node[sloped,above] {5} (C)
    (A) edge node[sloped,above] {19} (D)
    (B) edge[<-, bend right =-40] node[sloped,above] {2} (E)
    (B) edge[<-, bend right =-30] node[sloped,above] {3} (F)
    (C) edge[<-, bend right =-40] node[sloped,above] {1} (F)
    (C) edge[ultra thick] node[sloped,above] {5} (G)
    (C) edge[<-, bend right =-30] node[sloped,above] {4} (G)
    (D) edge node[sloped,above] {8} (G)
    (E) edge node[sloped,above] {3} (H)
    (E) edge[<-, bend right =-40] node[sloped,above] {2} (H)
    (F) edge[<-, bend right =-30] node[sloped,above] {3} (H)
    (F) edge node[sloped,above] {9} (I)
    (F) edge[<-, bend right =-40] node[sloped,above] {1} (I)
    (F) edge node[sloped,above] {8} (J)
    (G) edge[ultra thick] node[sloped,above] {2} (J)
    (G) edge[<-, bend left =-40] node[sloped,below] {4} (J)
    (H) edge node[sloped,above] {4} (K)
    (H) edge[<-, bend right =-40] node[sloped,above] {5} (K)
    (H) edge node[sloped,above] {5} (L)
    (I) edge node[sloped,above] {6} (L)
    (I) edge[<-, bend right =-40] node[sloped,above] {1} (L)
    (J) edge node[sloped,above] {4} (M)
    (J) edge[ultra thick] node[sloped,above] {0} (L)
    (J) edge[<-, bend right =-30] node[sloped,above] {1} (L)
    (K) edge node[sloped,above] {9} (N)
    (K) edge[<-, bend right =-40] node[sloped,above] {5} (N)
    (L) edge[ultra thick] node[sloped,above] {7} (N)
    (L) edge[<-, bend right =-40] node[sloped,above] {5} (N)
    (M) edge node[sloped,above] {10} (N);
    }\\\\
    ACGJMN - 2\\
    Остаточная сеть:\\
    \usetikzlibrary{graphs,automata,positioning}
    \tikz {
    	\path
    (2, 0) node (B) {B}
    (4, 0) node (E) {E}
    (6, 0) node (H) {H}
    (8, 0) node (K) {K}
    (0, -2) node (A) {A}
    (2, -2) node (C) {C}
    (4, -2) node (F) {F}
    (6, -2) node (I) {I}
    (8, -2) node (L) {L}
    (10, -2) node (N) {N}
    (2, -4) node (D) {D}
    (4, -4) node (G) {G}
    (6, -4) node (J) {J}
    (8, -4) node (M) {M};
    \path[->]
    (A) edge node[sloped,above] {17} (B)
    (A) edge[<-, bend right =-40] node[sloped,above] {5} (B)
    (A) edge[ultra thick] node[sloped,above] {2} (C)
    (A) edge[<-, bend right =-40] node[sloped,above] {7} (C)
    (A) edge node[sloped,above] {19} (D)
    
    (B) edge[<-, bend right =-40] node[sloped,above] {2} (E)
    (B) edge[<-, bend right =-30] node[sloped,above] {3} (F)
    
    (C) edge[<-, bend right =-40] node[sloped,above] {1} (F)
    (C) edge[ultra thick] node[sloped,above] {3} (G)
    (C) edge[<-, bend right =-30] node[sloped,above] {6} (G)
    
    (D) edge node[sloped,above] {8} (G)
    
    (E) edge node[sloped,above] {3} (H)
    (E) edge[<-, bend right =-40] node[sloped,above] {2} (H)
    
    (F) edge[<-, bend right =-30] node[sloped,above] {3} (H)
    (F) edge node[sloped,above] {9} (I)
    (F) edge[<-, bend right =-40] node[sloped,above] {1} (I)
    (F) edge node[sloped,above] {8} (J)
    
    (G) edge[ultra thick] node[sloped,above] {0} (J)
    (G) edge[<-, bend left =-40] node[sloped,below] {6} (J)
    
    (H) edge node[sloped,above] {4} (K)
    (H) edge[<-, bend right =-40] node[sloped,above] {5} (K)
    (H) edge node[sloped,above] {5} (L)
    
    (I) edge node[sloped,above] {6} (L)
    (I) edge[<-, bend right =-40] node[sloped,above] {1} (L)
    
    (J) edge[ultra thick] node[sloped,above] {2} (M)
    (J) edge[<-, bend left =-40] node[sloped,below] {2} (M)
    (J) edge[<-, bend right =-30] node[sloped,above] {1} (L)
    
    (K) edge node[sloped,above] {9} (N)
    (K) edge[<-, bend right =-40] node[sloped,above] {5} (N)
    
    (L) edge node[sloped,above] {7} (N)
    (L) edge[<-, bend right =-40] node[sloped,above] {5} (N)
    
    (M) edge[ultra thick] node[sloped,above] {8} (N)
    (M) edge[<-, bend left =-30] node[sloped,below] {2} (N);
    }\\\\
    \textbf{Ответ:} 2 + 3 + 1 + 4 + 2 = 12
\end{proof}

%% Условие задания 7
\begin{problem}
Найдите максимальный поток через данную сеть:
\begin{center}
    \usetikzlibrary{graphs,automata,positioning}
    \tikz {
    	\path
    (2, 0) node (B) {B}
    (4, 0) node (E) {E}
    (6, 0) node (H) {H}
    (8, 0) node (K) {K}
    (0, -2) node (A) {A}
    (2, -2) node (C) {C}
    (4, -2) node (F) {F}
    (6, -2) node (I) {I}
    (8, -2) node (L) {L}
    (10, -2) node (N) {N}
    (2, -4) node (D) {D}
    (4, -4) node (G) {G}
    (6, -4) node (J) {J}
    (8, -4) node (M) {M};
    \path[->]
    (A) edge node[sloped,above] {10} (B)
    (A) edge node[sloped,above] {20} (C)
    (A) edge node[sloped,above] {15} (D)
    
    (B) edge node[sloped,above] {3} (E)
    (B) edge node[sloped,above,pos=0.2] {5} (F)
    
    (C) edge node[sloped,above,pos=0.2] {5} (E)
    (C) edge node[sloped,above,pos=0.2] {9} (G)

    (D) edge node[sloped,above,pos=0.2] {10} (F)
    (D) edge node[sloped,above] {6} (G)
    
    (E) edge node[sloped,above] {5} (H)
    (E) edge node[sloped,above,pos=0.2] {1} (I)
    
    (F) edge node[sloped,above,pos=0.2] {10} (H)
    (F) edge node[sloped,above,pos=0.2] {4} (J)
    
    (G) edge node[sloped,above,pos=0.2] {4} (J)
    (G) edge node[sloped,above,pos=0.2] {7} (I)
    
    (H) edge node[sloped,above] {6} (K)
    (H) edge node[sloped,above,pos=0.2] {10} (L)
    
    (I) edge node[sloped,above,pos=0.2] {8} (K)
    (I) edge node[sloped,above,pos=0.2] {7} (M)
    
    (J) edge node[sloped,above,pos=0.2] {5} (L)
    (J) edge node[sloped,above] {9} (M)
    
    (K) edge node[sloped,above] {21} (N)
    
    (L) edge node[sloped,above] {20} (N)
    
    (M) edge node[sloped,above] {20} (N);
    }
\end{center}
\end{problem}
%% Решение задания 7
\begin{proof} $ $\\
ACGIMN - 7\\
Остаточная сеть:\\
        \usetikzlibrary{graphs,automata,positioning}
    \tikz {
    	\path
    (2, 0) node (B) {B}
    (4, 0) node (E) {E}
    (6, 0) node (H) {H}
    (8, 0) node (K) {K}
    (0, -2) node (A) {A}
    (2, -2) node (C) {C}
    (4, -2) node (F) {F}
    (6, -2) node (I) {I}
    (8, -2) node (L) {L}
    (10, -2) node (N) {N}
    (2, -4) node (D) {D}
    (4, -4) node (G) {G}
    (6, -4) node (J) {J}
    (8, -4) node (M) {M};
    \path[->]
    (A) edge node[sloped,above] {10} (B)
    (A) edge[ultra thick] node[sloped,above] {13} (C)
    (A) edge[<-, bend right =-40] node[sloped,above] {7} (C)
    (A) edge node[sloped,above] {15} (D)
    
    (B) edge node[sloped,above] {3} (E)
    (B) edge node[sloped,above,pos=0.2] {5} (F)
    
    (C) edge node[sloped,above,pos=0.2] {5} (E)
    (C) edge[ultra thick] node[sloped,above,pos=0.2] {2} (G)
    (C) edge[<-, bend right =-30] node[sloped,above,pos=0.7] {7} (G)

    (D) edge node[sloped,above,pos=0.2] {10} (F)
    (D) edge node[sloped,above] {6} (G)
    
    (E) edge node[sloped,above] {5} (H)
    (E) edge node[sloped,above,pos=0.2] {1} (I)
    
    (F) edge node[sloped,above,pos=0.2] {10} (H)
    (F) edge node[sloped,above,pos=0.2] {4} (J)
    
    (G) edge node[sloped,above,pos=0.2] {4} (J)
    (G) edge[ultra thick] node[sloped,above,pos=0.2] {0} (I)
    (G) edge[<-, bend right =-30] node[sloped,above,pos=0.8] {7} (I)
    
    (H) edge node[sloped,above] {6} (K)
    (H) edge node[sloped,above,pos=0.2] {10} (L)
    
    (I) edge node[sloped,above,pos=0.2] {8} (K)
    (I) edge[ultra thick] node[sloped,above,pos=0.2] {0} (M)
    (I) edge[<-, bend right =-40] node[sloped,above,pos=0.2] {7} (M)
    
    (J) edge node[sloped,above,pos=0.2] {5} (L)
    (J) edge node[sloped,above] {9} (M)
    
    (K) edge node[sloped,above] {21} (N)
    
    (L) edge node[sloped,above] {20} (N)
    
    (M) edge[ultra thick] node[sloped,above] {13} (N)
    (M) edge[<-, bend left =-40] node[sloped,below] {7} (N);
    }\\\\
ADHLN - 10\\
    Остаточная сеть:\\
        \usetikzlibrary{graphs,automata,positioning}
    \tikz {
    	\path
    (2, 0) node (B) {B}
    (4, 0) node (E) {E}
    (6, 0) node (H) {H}
    (8, 0) node (K) {K}
    (0, -2) node (A) {A}
    (2, -2) node (C) {C}
    (4, -2) node (F) {F}
    (6, -2) node (I) {I}
    (8, -2) node (L) {L}
    (10, -2) node (N) {N}
    (2, -4) node (D) {D}
    (4, -4) node (G) {G}
    (6, -4) node (J) {J}
    (8, -4) node (M) {M};
    \path[->]
    (A) edge node[sloped,above] {10} (B)
    (A) edge node[sloped,above] {13} (C)
    (A) edge[<-, bend right =-40] node[sloped,above] {7} (C)
    (A) edge[ultra thick] node[sloped,above] {5} (D)
    (A) edge[<-, bend left =-40] node[sloped,below] {10} (D)
    
    (B) edge node[sloped,above] {3} (E)
    (B) edge node[sloped,above,pos=0.2] {5} (F)
    
    (C) edge node[sloped,above,pos=0.2] {5} (E)
    (C) edge node[sloped,above,pos=0.2] {2} (G)
    (C) edge[<-, bend right =-30] node[sloped,above,pos=0.7] {7} (G)

    (D) edge[ultra thick] node[sloped,above,pos=0.2] {0} (F)
    (D) edge[<-, bend right =-30] node[sloped,above,pos=0.8] {10} (F)
    (D) edge node[sloped,above] {6} (G)
    
    (E) edge node[sloped,above] {5} (H)
    (E) edge node[sloped,above,pos=0.2] {1} (I)
    
    (F) edge[ultra thick] node[sloped,above,pos=0.2] {0} (H)
    (F) edge[<-, bend right =-30] node[sloped,above,pos=0.2] {10} (H)
    (F) edge node[sloped,above,pos=0.2] {4} (J)
    
    (G) edge node[sloped,above,pos=0.2] {4} (J)
    (G) edge[<-, bend right =-30] node[sloped,above,pos=0.8] {7} (I)
    
    (H) edge node[sloped,above] {6} (K)
    (H) edge[ultra thick] node[sloped,above,pos=0.2] {0} (L)
    (H) edge[<-, bend right =-30] node[sloped,above,pos=0.7] {10} (L)
    
    (I) edge node[sloped,above,pos=0.2] {8} (K)
    (I) edge[<-, bend right =-40] node[sloped,above,pos=0.2] {7} (M)
    
    (J) edge node[sloped,above,pos=0.2] {5} (L)
    (J) edge node[sloped,above] {9} (M)
    
    (K) edge node[sloped,above] {21} (N)
    
    (L) edge[ultra thick] node[sloped,above] {10} (N)
    (L) edge[<-, bend right =-30] node[sloped,above,pos=0.2] {10} (N)
    
    (M) edge node[sloped,above] {13} (N)
    (M) edge[<-, bend left =-40] node[sloped,below] {7} (N);
    }\\\\
ADGJMN - 4\\
    Остаточная сеть:\\
        \usetikzlibrary{graphs,automata,positioning}
    \tikz {
    	\path
    (2, 0) node (B) {B}
    (4, 0) node (E) {E}
    (6, 0) node (H) {H}
    (8, 0) node (K) {K}
    (0, -2) node (A) {A}
    (2, -2) node (C) {C}
    (4, -2) node (F) {F}
    (6, -2) node (I) {I}
    (8, -2) node (L) {L}
    (10, -2) node (N) {N}
    (2, -4) node (D) {D}
    (4, -4) node (G) {G}
    (6, -4) node (J) {J}
    (8, -4) node (M) {M};
    \path[->]
    (A) edge node[sloped,above] {10} (B)
    (A) edge node[sloped,above] {13} (C)
    (A) edge[<-, bend right =-40] node[sloped,above] {7} (C)
    (A) edge[ultra thick] node[sloped,above] {1} (D)
    (A) edge[<-, bend left =-40] node[sloped,below] {14} (D)
    
    (B) edge node[sloped,above] {3} (E)
    (B) edge node[sloped,above,pos=0.2] {5} (F)
    
    (C) edge node[sloped,above,pos=0.2] {5} (E)
    (C) edge node[sloped,above,pos=0.2] {2} (G)
    (C) edge[<-, bend right =-30] node[sloped,above,pos=0.7] {7} (G)

    (D) edge[<-, bend right =-30] node[sloped,above,pos=0.8] {10} (F)
    (D) edge[ultra thick] node[sloped,above] {2} (G)
    (D) edge[<-, bend left =-40] node[sloped,below] {4} (G)
    
    (E) edge node[sloped,above] {5} (H)
    (E) edge node[sloped,above,pos=0.2] {1} (I)
    
    (F) edge[<-, bend right =-30] node[sloped,above,pos=0.2] {10} (H)
    (F) edge node[sloped,above,pos=0.2] {4} (J)
    
    (G) edge[ultra thick] node[sloped,above,pos=0.2] {0} (J)
    (G) edge[<-, bend right =-30] node[sloped,above,pos=0.8] {7} (I)
    (G) edge[<-, bend left =-40] node[sloped,below] {4} (J)
    
    (H) edge node[sloped,above] {6} (K)
    (H) edge[<-, bend right =-30] node[sloped,above,pos=0.7] {10} (L)
    
    (I) edge node[sloped,above,pos=0.2] {8} (K)
    (I) edge[<-, bend right =-40] node[sloped,above,pos=0.2] {7} (M)
    
    (J) edge node[sloped,above,pos=0.2] {5} (L)
    (J) edge[ultra thick] node[sloped,above] {5} (M)
    (J) edge[<-, bend left =-40] node[sloped,below] {4} (M)
    
    (K) edge node[sloped,above] {21} (N)
    
    (L) edge node[sloped,above] {10} (N)
    (L) edge[<-, bend right =-30] node[sloped,above,pos=0.2] {10} (N)
    
    (M) edge[ultra thick] node[sloped,above] {9} (N)
    (M) edge[<-, bend left =-40] node[sloped,below] {11} (N);
    }\\\\
ABEHKN - 3\\
    Остаточная сеть:\\
        \usetikzlibrary{graphs,automata,positioning}
    \tikz {
    	\path
    (2, 0) node (B) {B}
    (4, 0) node (E) {E}
    (6, 0) node (H) {H}
    (8, 0) node (K) {K}
    (0, -2) node (A) {A}
    (2, -2) node (C) {C}
    (4, -2) node (F) {F}
    (6, -2) node (I) {I}
    (8, -2) node (L) {L}
    (10, -2) node (N) {N}
    (2, -4) node (D) {D}
    (4, -4) node (G) {G}
    (6, -4) node (J) {J}
    (8, -4) node (M) {M};
    \path[->]
    (A) edge[ultra thick] node[sloped,above] {7} (B)
    (A) edge[<-, bend right =-40] node[sloped,above] {3} (B)
    (A) edge node[sloped,above] {13} (C)
    (A) edge[<-, bend right =-40] node[sloped,above] {7} (C)
    (A) edge node[sloped,above] {1} (D)
    (A) edge[<-, bend left =-40] node[sloped,below] {14} (D)
    
    (B) edge[ultra thick] node[sloped,above] {0} (E)
    (B) edge[<-, bend right =-40] node[sloped,above] {3} (E)
    (B) edge node[sloped,above,pos=0.2] {5} (F)
    
    (C) edge node[sloped,above,pos=0.2] {5} (E)
    (C) edge node[sloped,above,pos=0.2] {2} (G)
    (C) edge[<-, bend right =-30] node[sloped,above,pos=0.7] {7} (G)

    (D) edge[<-, bend right =-30] node[sloped,above,pos=0.8] {10} (F)
    (D) edge node[sloped,above] {2} (G)
    (D) edge[<-, bend left =-40] node[sloped,below] {4} (G)
    
    (E) edge[ultra thick] node[sloped,above] {2} (H)
    (E) edge[<-, bend right =-40] node[sloped,above] {3} (H)
    (E) edge node[sloped,above,pos=0.2] {1} (I)
    
    (F) edge[<-, bend right =-30] node[sloped,above,pos=0.2] {10} (H)
    (F) edge node[sloped,above,pos=0.2] {4} (J)
    
    (G) edge[<-, bend right =-30] node[sloped,above,pos=0.8] {7} (I)
    (G) edge[<-, bend left =-40] node[sloped,below] {4} (J)
    
    (H) edge[ultra thick] node[sloped,above] {3} (K)
    (H) edge[<-, bend right =-40] node[sloped,above] {3} (K)
    (H) edge[<-, bend right =-30] node[sloped,above,pos=0.7] {10} (L)
    
    (I) edge node[sloped,above,pos=0.2] {8} (K)
    (I) edge[<-, bend right =-40] node[sloped,above,pos=0.2] {7} (M)
    
    (J) edge node[sloped,above,pos=0.2] {5} (L)
    (J) edge node[sloped,above] {5} (M)
    (J) edge[<-, bend left =-40] node[sloped,below] {4} (M)
    
    (K) edge[ultra thick] node[sloped,above] {18} (N)
    (K) edge[<-, bend right =-40] node[sloped,above] {3} (N)
    
    (L) edge node[sloped,above] {10} (N)
    (L) edge[<-, bend right =-30] node[sloped,above,pos=0.2] {10} (N)
    
    (M) edge node[sloped,above] {9} (N)
    (M) edge[<-, bend left =-40] node[sloped,below] {11} (N);
    }\\\\
ABFJMN - 4\\
    Остаточная сеть:\\
        \usetikzlibrary{graphs,automata,positioning}
    \tikz {
    	\path
    (2, 0) node (B) {B}
    (4, 0) node (E) {E}
    (6, 0) node (H) {H}
    (8, 0) node (K) {K}
    (0, -2) node (A) {A}
    (2, -2) node (C) {C}
    (4, -2) node (F) {F}
    (6, -2) node (I) {I}
    (8, -2) node (L) {L}
    (10, -2) node (N) {N}
    (2, -4) node (D) {D}
    (4, -4) node (G) {G}
    (6, -4) node (J) {J}
    (8, -4) node (M) {M};
    \path[->]
    (A) edge[ultra thick] node[sloped,above] {3} (B)
    (A) edge[<-, bend right =-40] node[sloped,above] {7} (B)
    (A) edge node[sloped,above] {13} (C)
    (A) edge[<-, bend right =-40] node[sloped,above] {7} (C)
    (A) edge node[sloped,above] {1} (D)
    (A) edge[<-, bend left =-40] node[sloped,below] {14} (D)
    
    (B) edge[<-, bend right =-40] node[sloped,above] {3} (E)
    (B) edge[ultra thick] node[sloped,above,pos=0.2] {1} (F)
    (B) edge[<-, bend right =-30] node[sloped,above,pos=0.6] {4} (F)
    
    (C) edge node[sloped,above,pos=0.2] {5} (E)
    (C) edge node[sloped,above,pos=0.2] {2} (G)
    (C) edge[<-, bend right =-30] node[sloped,above,pos=0.7] {7} (G)

    (D) edge[<-, bend right =-30] node[sloped,above,pos=0.8] {10} (F)
    (D) edge node[sloped,above] {2} (G)
    (D) edge[<-, bend left =-40] node[sloped,below] {4} (G)
    
    (E) edge node[sloped,above] {2} (H)
    (E) edge[<-, bend right =-40] node[sloped,above] {3} (H)
    (E) edge node[sloped,above,pos=0.2] {1} (I)
    
    (F) edge[<-, bend right =-30] node[sloped,above,pos=0.2] {10} (H)
    (F) edge[ultra thick] node[sloped,above,pos=0.2] {0} (J)
    (F) edge[<-, bend right =-30] node[sloped,above,pos=0.6] {4} (J)
    
    (G) edge[<-, bend right =-30] node[sloped,above,pos=0.8] {7} (I)
    (G) edge[<-, bend left =-40] node[sloped,below] {4} (J)
    
    (H) edge node[sloped,above] {3} (K)
    (H) edge[<-, bend right =-40] node[sloped,above] {3} (K)
    (H) edge[<-, bend right =-30] node[sloped,above,pos=0.7] {10} (L)
    
    (I) edge node[sloped,above,pos=0.2] {8} (K)
    (I) edge[<-, bend right =-40] node[sloped,above,pos=0.2] {7} (M)
    
    (J) edge node[sloped,above,pos=0.2] {5} (L)
    (J) edge[ultra thick] node[sloped,above] {1} (M)
    (J) edge[<-, bend left =-40] node[sloped,below] {8} (M)
    
    (K) edge node[sloped,above] {18} (N)
    (K) edge[<-, bend right =-40] node[sloped,above] {3} (N)
    
    (L) edge node[sloped,above] {10} (N)
    (L) edge[<-, bend right =-30] node[sloped,above,pos=0.2] {10} (N)
    
    (M) edge[ultra thick] node[sloped,above] {5} (N)
    (M) edge[<-, bend left =-40] node[sloped,below] {15} (N);
    }\\\\
ACEHKN - 2\\
    Остаточная сеть:\\
        \usetikzlibrary{graphs,automata,positioning}
    \tikz {
    	\path
    (2, 0) node (B) {B}
    (4, 0) node (E) {E}
    (6, 0) node (H) {H}
    (8, 0) node (K) {K}
    (0, -2) node (A) {A}
    (2, -2) node (C) {C}
    (4, -2) node (F) {F}
    (6, -2) node (I) {I}
    (8, -2) node (L) {L}
    (10, -2) node (N) {N}
    (2, -4) node (D) {D}
    (4, -4) node (G) {G}
    (6, -4) node (J) {J}
    (8, -4) node (M) {M};
    \path[->]
    (A) edge node[sloped,above] {3} (B)
    (A) edge[<-, bend right =-40] node[sloped,above] {7} (B)
    (A) edge[ultra thick] node[sloped,above] {11} (C)
    (A) edge[<-, bend right =-40] node[sloped,above] {9} (C)
    (A) edge node[sloped,above] {1} (D)
    (A) edge[<-, bend left =-40] node[sloped,below] {14} (D)
    
    (B) edge[<-, bend right =-40] node[sloped,above] {3} (E)
    (B) edge node[sloped,above,pos=0.2] {1} (F)
    (B) edge[<-, bend right =-30] node[sloped,above,pos=0.6] {4} (F)
    
    (C) edge[ultra thick] node[sloped,above,pos=0.2] {3} (E)
    (C) edge[<-, bend right =-30] node[sloped,above,pos=0.2] {2} (E)
    (C) edge node[sloped,above,pos=0.2] {2} (G)
    (C) edge[<-, bend right =-30] node[sloped,above,pos=0.7] {7} (G)

    (D) edge[<-, bend right =-30] node[sloped,above,pos=0.8] {10} (F)
    (D) edge node[sloped,above] {2} (G)
    (D) edge[<-, bend left =-40] node[sloped,below] {4} (G)
    
    (E) edge[ultra thick] node[sloped,above] {0} (H)
    (E) edge[<-, bend right =-40] node[sloped,above] {5} (H)
    (E) edge node[sloped,above,pos=0.2] {1} (I)
    
    (F) edge[<-, bend right =-30] node[sloped,above,pos=0.2] {10} (H)
    (F) edge[<-, bend right =-30] node[sloped,above,pos=0.6] {4} (J)
    
    (G) edge[<-, bend right =-30] node[sloped,above,pos=0.8] {7} (I)
    (G) edge[<-, bend left =-40] node[sloped,below] {4} (J)
    
    (H) edge[ultra thick] node[sloped,above] {1} (K)
    (H) edge[<-, bend right =-40] node[sloped,above] {5} (K)
    (H) edge[<-, bend right =-30] node[sloped,above,pos=0.7] {10} (L)
    
    (I) edge node[sloped,above,pos=0.2] {8} (K)
    (I) edge[<-, bend right =-40] node[sloped,above,pos=0.2] {7} (M)
    
    (J) edge node[sloped,above,pos=0.2] {5} (L)
    (J) edge node[sloped,above] {1} (M)
    (J) edge[<-, bend left =-40] node[sloped,below] {8} (M)
    
    (K) edge[ultra thick] node[sloped,above] {16} (N)
    (K) edge[<-, bend right =-40] node[sloped,above] {5} (N)
    
    (L) edge node[sloped,above] {10} (N)
    (L) edge[<-, bend right =-30] node[sloped,above,pos=0.2] {10} (N)
    
    (M) edge node[sloped,above] {5} (N)
    (M) edge[<-, bend left =-40] node[sloped,below] {15} (N);
    }\\\\
ACEIKN - 1\\
    Остаточная сеть:\\
        \usetikzlibrary{graphs,automata,positioning}
    \tikz {
    	\path
    (2, 0) node (B) {B}
    (4, 0) node (E) {E}
    (6, 0) node (H) {H}
    (8, 0) node (K) {K}
    (0, -2) node (A) {A}
    (2, -2) node (C) {C}
    (4, -2) node (F) {F}
    (6, -2) node (I) {I}
    (8, -2) node (L) {L}
    (10, -2) node (N) {N}
    (2, -4) node (D) {D}
    (4, -4) node (G) {G}
    (6, -4) node (J) {J}
    (8, -4) node (M) {M};
    \path[->]
    (A) edge node[sloped,above] {3} (B)
    (A) edge[<-, bend right =-40] node[sloped,above] {7} (B)
    (A) edge[ultra thick] node[sloped,above] {10} (C)
    (A) edge[<-, bend right =-40] node[sloped,above] {10} (C)
    (A) edge node[sloped,above] {1} (D)
    (A) edge[<-, bend left =-40] node[sloped,below] {14} (D)
    
    (B) edge[<-, bend right =-40] node[sloped,above] {3} (E)
    (B) edge node[sloped,above,pos=0.2] {1} (F)
    (B) edge[<-, bend right =-30] node[sloped,above,pos=0.6] {4} (F)
    
    (C) edge[ultra thick] node[sloped,above,pos=0.2] {2} (E)
    (C) edge[<-, bend right =-30] node[sloped,above,pos=0.2] {3} (E)
    (C) edge node[sloped,above,pos=0.2] {2} (G)
    (C) edge[<-, bend right =-30] node[sloped,above,pos=0.7] {7} (G)

    (D) edge[<-, bend right =-30] node[sloped,above,pos=0.8] {10} (F)
    (D) edge node[sloped,above] {2} (G)
    (D) edge[<-, bend left =-40] node[sloped,below] {4} (G)
    
    (E) edge[<-, bend right =-40] node[sloped,above] {5} (H)
    (E) edge[ultra thick] node[sloped,above,pos=0.2] {0} (I)
    (E) edge[<-, bend right =-30] node[sloped,above,pos=0.7] {1} (I)
    
    (F) edge[<-, bend right =-30] node[sloped,above,pos=0.2] {10} (H)
    (F) edge[<-, bend right =-30] node[sloped,above,pos=0.6] {4} (J)
    
    (G) edge[<-, bend right =-30] node[sloped,above,pos=0.8] {7} (I)
    (G) edge[<-, bend left =-40] node[sloped,below] {4} (J)
    
    (H) edge node[sloped,above] {1} (K)
    (H) edge[<-, bend right =-40] node[sloped,above] {5} (K)
    (H) edge[<-, bend right =-30] node[sloped,above,pos=0.7] {10} (L)
    
    (I) edge[ultra thick] node[sloped,above,pos=0.2] {7} (K)
    (I) edge[<-, bend right =-30] node[sloped,above,pos=0.7] {1} (K)
    (I) edge[<-, bend right =-40] node[sloped,above,pos=0.2] {8} (M)
    
    (J) edge node[sloped,above,pos=0.2] {5} (L)
    (J) edge node[sloped,above] {1} (M)
    (J) edge[<-, bend left =-40] node[sloped,below] {8} (M)
    
    (K) edge[ultra thick] node[sloped,above] {15} (N)
    (K) edge[<-, bend right =-40] node[sloped,above] {6} (N)
    
    (L) edge node[sloped,above] {10} (N)
    (L) edge[<-, bend right =-30] node[sloped,above,pos=0.2] {10} (N)
    
    (M) edge node[sloped,above] {5} (N)
    (M) edge[<-, bend left =-40] node[sloped,below] {15} (N);
    }\\\\
Максимальный поток в этой сети равен 31. Проверим это с помощью пропускной способности ребер, составляющих минимальный разрез. Минимальный разрез для данного графа = \{EH, EI, FH, FJ, GI, GJ\}, пропускная способность которых также равна 31.\\
    \usetikzlibrary{graphs,automata,positioning}
    \tikz {
    	\path
    (2, 0) node (B) {B}
    (4, 0) node (E) {E}
    (6, 0) node (H) {H}
    (8, 0) node (K) {K}
    (0, -2) node (A) {A}
    (2, -2) node (C) {C}
    (4, -2) node (F) {F}
    (6, -2) node (I) {I}
    (8, -2) node (L) {L}
    (10, -2) node (N) {N}
    (2, -4) node (D) {D}
    (4, -4) node (G) {G}
    (6, -4) node (J) {J}
    (8, -4) node (M) {M};
    \path[->]
    (A) edge node[sloped,above] {10} (B)
    (A) edge node[sloped,above] {20} (C)
    (A) edge node[sloped,above] {15} (D)
    
    (B) edge[dashed] node[sloped,above] {3} (E)
    (B) edge node[sloped,above,pos=0.2] {5} (F)
    
    (C) edge node[sloped,above,pos=0.2] {5} (E)
    (C) edge node[sloped,above,pos=0.2] {9} (G)

    (D) edge node[sloped,above,pos=0.2] {10} (F)
    (D) edge node[sloped,above] {6} (G)
    
    (E) edge[dashed] node[sloped,above] {5} (H)
    (E) edge[dashed] node[sloped,above,pos=0.2] {1} (I)
    
    (F) edge[dashed] node[sloped,above,pos=0.2] {10} (H)
    (F) edge[dashed] node[sloped,above,pos=0.2] {4} (J)
    
    (G) edge[dashed] node[sloped,above,pos=0.2] {4} (J)
    (G) edge[dashed] node[sloped,above,pos=0.2] {7} (I)
    
    (H) edge node[sloped,above] {6} (K)
    (H) edge node[sloped,above,pos=0.2] {10} (L)
    
    (I) edge node[sloped,above,pos=0.2] {8} (K)
    (I) edge[dashed] node[sloped,above,pos=0.2] {7} (M)
    
    (J) edge node[sloped,above,pos=0.2] {5} (L)
    (J) edge node[sloped,above] {9} (M)
    
    (K) edge node[sloped,above] {21} (N)
    
    (L) edge node[sloped,above] {20} (N)
    
    (M) edge node[sloped,above] {20} (N);
    }\\
    \textbf{Ответ:} 31
\end{proof}
%% Условие задания 8
\begin{problem}
Найдите наибольшее паросочетание в двудольном графе, заданном набором рёбер (a, $\gamma$) (a, $\delta$) (a, $\epsilon$) (b, $\beta$) (b, $\theta$) (c, $\theta$) (d, $\delta$) (d, $\zeta$) (d, $\theta$) (e, $\alpha$) (e, $\beta$) (e, $\eta$) (f, $\zeta$) (g, $\gamma$) (g, $\delta$) (h, $\gamma$)
\end{problem}
%% Решение задания 8
\begin{proof} $ $\\
Изначальный граф:
\begin{center}
\tikz {
        \path
(0, -7) node (S) {S}
(2, 0) node (a) {a}
(8, 0) node (1) {$\gamma$}
(2, -2) node (g) {g}
(8, -2) node (2) {$\delta$}
(2, -4) node (h) {h}
(8, -4) node (3) {$\varepsilon$}
(2, -6) node (b) {b}
(8, -6) node (4) {$\beta$}
(2, -8) node (c) {c}
(8, -8) node (5) {$\theta$}
(2, -10) node (d) {d}
(8, -10) node (6) {$\zeta$}
(2, -12) node (e) {e}
(8, -12) node (7) {$\alpha$}
(2, -14) node (f) {f}
(8, -14) node (8) {$\eta$}
(10, -7) node (F) {F};
\draw[->] (S) -- (a);
\draw[->] (S) -- (g);
\draw[->] (S) -- (h);
\draw[->] (S) -- (b);
\draw[->] (S) -- (c);
\draw[->] (S) -- (d);
\draw[->] (S) -- (e);
\draw[->] (S) -- (f);

\draw[<-] (F) -- (1);
\draw[<-] (F) -- (2);
\draw[<-] (F) -- (3);
\draw[<-] (F) -- (4);
\draw[<-] (F) -- (5);
\draw[<-] (F) -- (6);
\draw[<-] (F) -- (7);
\draw[<-] (F) -- (8);

\draw[->] (a) -- (1);
\draw[->] (a) -- (2);
\draw[->] (a) -- (3);
\draw[->] (g) -- (1);
\draw[->] (g) -- (2);
\draw[->] (h) -- (1);
\draw[->] (b) -- (4);
\draw[->] (b) -- (5);
\draw[->] (c) -- (5);
\draw[->] (d) -- (2);
\draw[->] (d) -- (5);
\draw[->] (d) -- (6);
\draw[->] (e) -- (4);
\draw[->] (e) -- (7);
\draw[->] (e) -- (8);
\draw[->] (f) -- (6);
}
\end{center} $ $\\
Ответ: (a, $\eta$) (b, $\beta$) (c, $\alpha$)
(d, $\varepsilon$) (e, $\theta$) (f, $\delta$)  (g, $\zeta$) (h, $\gamma$)

Пройденные пути:\\
\{S, a, $\gamma$, F\}, \{S, g, $\delta$, F\}, \{S, h, $\gamma$, a, $\varepsilon$, F\}, \{S, b, $\beta$, F\}, \{S, c, $\theta$, F\}, \{S, d, $\zeta$, F\}, \{S, e, $\alpha$, F\}

Граф после применения алгоритма:
\begin{center}
% \usetikzlibrary{graphs,automata,positioning}
\tikz {
        \path
(0, -7) node (S) {S}
(2, 0) node (a) {a}
(8, 0) node (1) {$\gamma$}
(2, -2) node (g) {g}
(8, -2) node (2) {$\delta$}
(2, -4) node (h) {h}
(8, -4) node (3) {$\varepsilon$}
(2, -6) node (b) {b}
(8, -6) node (4) {$\beta$}
(2, -8) node (c) {c}
(8, -8) node (5) {$\theta$}
(2, -10) node (d) {d}
(8, -10) node (6) {$\zeta$}
(2, -12) node (e) {e}
(8, -12) node (7) {$\alpha$}
(2, -14) node (f) {f}
(8, -14) node (8) {$\eta$}
(10, -7) node (F) {F};

\draw[<-] (S) -- (f);
\draw[<-] (F) -- (8);

\draw[->] (a) -- (1);
\draw[->] (a) -- (2);
\draw[<-, red] (a) -- (3);
\draw[->] (g) -- (1);
\draw[<-, red] (g) -- (2);
\draw[<-, red] (h) -- (1);
\draw[<-, red] (b) -- (4);
\draw[->] (b) -- (5);
\draw[<-, red] (c) -- (5);
\draw[->] (d) -- (2);
\draw[->] (d) -- (5);
\draw[<-, red] (d) -- (6);
\draw[->] (e) -- (4);
\draw[<-, red] (e) -- (7);
\draw[->] (e) -- (8);
\draw[->] (f) -- (6);
}
\end{center} $ $\\
Ответ: $\tilde{E} = \{a\varepsilon,  g\delta,  h\gamma,  b\beta,  c\theta,  d\zeta,  e\alpha\}$
\end{proof}

%% Условие задания 9
\begin{problem}
Найдите радиус, диаметр и центр данного дерева:
\begin{center}
\begin{tikzcd}
            1 \ar[r, dash] \ar[d, dash]
            &2 \ar[d, dash]
            &3 \ar[r, dash]
            &4 \ar[lld, dash]
            &5 \ar[llld, dash] \ar[lld, dash] \ar[ld, dash] \ar[d, dash]
            &6 \ar[l, dash]
            \\
            7 
            &8 
            &9 
            &10
            &11
        \end{tikzcd}
\end{center}
\end{problem}
%% Решение задания 9
\begin{proof} $ $ \\
\begin{center}
\bordermatrix{ & 1 & 2 & 3 & 4 & 5 & 6 & 7 & 8 & 9 & 10 & 11 & max \cr
             1 & 0 & 1 & 4 & 3 & 3 & 4 & 1 & 2 & 4 & 4 & 4 & 4 \cr
             2 & 1 & 0 & 3 & 2 & 2 & 3 & 2 & 1 & 3 & 3 & 3 & 3 \cr
             3 & 4 & 3 & 0 & 1 & 3 & 4 & 5 & 2 & 4 & 4 & 4 & 5 \cr
             4 & 3 & 2 & 1 & 0 & 2 & 3 & 4 & 1 & 3 & 3 & 3 & 4 \cr
             5 & 3 & 2 & 3 & 2 & 0 & 1 & 4 & 1 & 1 & 1 & 1 & 4 \cr
             6 & 4 & 3 & 4 & 3 & 1 & 0 & 5 & 2 & 2 & 2 & 2 & 5 \cr
             7 & 1 & 2 & 5 & 4 & 4 & 5 & 0 & 3 & 5 & 5 & 5 & 5 \cr
             8 & 2 & 1 & 2 & 1 & 1 & 2 & 3 & 0 & 2 & 2 & 2 & 3 \cr
             9 & 4 & 3 & 4 & 3 & 1 & 2 & 5 & 2 & 0 & 2 & 2 & 5 \cr
             10 & 4 & 3 & 4 & 3 & 1 & 2 & 5 & 2 & 2 & 0 & 2 & 5 \cr
             11 & 4 & 3 & 4 & 3 & 1 & 2 & 5 & 2 & 2 & 2 & 0 & 5 \cr}
                 \end{center}
радиус: 4\\
диаметр: 5\\
центры: 1, 4, 5
\end{proof}

%% Условие задания 10
$ $\\
\begin{problem}
Найдите радиус, диаметр и центр данного графа:
\begin{center}
\begin{tikzcd}
            A \ar[r, dash] \ar[d, dash]
            &B \ar[r, dash]
            &C \ar[r, dash] \ar[lddd, dash]
            &D \ar[d, dash] \ar[lddd, dash] \ar[llddd, dash]
            \\
            E \ar[d, dash] \ar[rdd, dash]
            &
            &
            &F \ar[d, dash]
            \\
            G \ar[d, dash] \ar[rd, dash]
            &
            &
            &H \ar[d, dash]
            \\
            I \ar[r, dash]
            & J \ar[r, dash]
            & K \ar[r, dash]
            & L
        \end{tikzcd}
\end{center}
\end{problem}
%% Решение задания 10
\begin{proof} $ $ \\
\begin{center}
\bordermatrix{ & A & B & C & D & E & F & G & H & I & J & K & L & max \cr
             A & 0 & 1 & 2 & 3 & 1 & 4 & 2 & 5 & 3 & 2 & 3 & 4 & 5 \cr
             B & 1 & 0 & 1 & 2 & 2 & 3 & 3 & 4 & 3 & 2 & 3 & 4 & 4 \cr
             C & 2 & 1 & 0 & 1 & 2 & 2 & 2 & 3 & 2 & 1 & 2 & 3 & 3 \cr
             D & 3 & 2 & 1 & 0 & 2 & 1 & 2 & 2 & 2 & 1 & 1 & 2 & 3 \cr
             E & 1 & 2 & 2 & 2 & 0 & 3 & 1 & 4 & 2 & 1 & 2 & 3 & 4 \cr
             F & 4 & 3 & 2 & 1 & 3 & 0 & 3 & 1 & 3 & 2 & 2 & 2 & 4 \cr
             G & 2 & 3 & 2 & 2 & 1 & 3 & 0 & 4 & 1 & 1 & 2 & 3 & 4 \cr
             H & 5 & 4 & 3 & 2 & 4 & 1 & 4 & 0 & 4 & 3 & 2 & 1 & 5 \cr
             I & 3 & 3 & 2 & 2 & 2 & 3 & 1 & 4 & 0 & 1 & 2 & 3 & 4 \cr
             J & 2 & 2 & 1 & 1 & 1 & 2 & 1 & 3 & 1 & 0 & 1 & 2 & 3 \cr
             K & 3 & 3 & 2 & 1 & 2 & 2 & 2 & 2 & 2 & 1 & 0 & 1 & 3 \cr
             L & 4 & 4 & 3 & 2 & 3 & 2 & 3 & 1 & 3 & 2 & 1 & 0 & 4 \cr}
\end{center}
радиус: 3\\
диаметр: 5\\
центры: C, D, J, K
\end{proof}

%% Условие задания 11
\begin{problem}
Постройте пример графа, для которого хроматическим многочленом является\\
$t^4(t-1)^8(t-2)^3(t-3)^2$
\end{problem}
%% Решение задания 11
\begin{proof} $ $\\
$t^2(t-1)^2(t-2)^2(t-3)^2$\\\\
\tikz {
        \path
        (0, 0) node[circle, fill=black] (1) {}
        (2, 0) node[circle, fill=black] (2) {}
        (0, -2) node[circle, fill=black] (3) {}
        (2, -2) node[circle, fill=black] (4) {}

        (4, 0) node[circle, fill=black] (5) {}
        (6, 0) node[circle, fill=black] (6) {}
        (4, -2) node[circle, fill=black] (7) {}
        (6, -2) node[circle, fill=black] (8) {};

        \draw (1) -- (2);
        \draw (1) -- (3);
        \draw (1) -- (4);
        \draw (2) -- (3);
        \draw (2) -- (4);
        \draw (3) -- (4);

        \draw (5) -- (6);
        \draw (5) -- (7);
        \draw (5) -- (8);
        \draw (6) -- (7);
        \draw (6) -- (8);
        \draw (7) -- (8);
        }$ $\\\\\\
$t(t-1)^3$\\\\
\tikz {
        \path
        (0, 0) node[circle, fill=black] (1) {}
        (2, 0) node[circle, fill=black] (2) {}
        (4, 0) node[circle, fill=black] (3) {}
        (6, 0) node[circle, fill=black] (4) {};

        \draw (1) -- (2);
        \draw (2) -- (3);
        \draw (3) -- (4);
        }$ $\\\\\\
$t(t-1)^3(t-2)$\\\\
\tikz {
        \path
        (0, 0) node[circle, fill=black] (1) {}
        (2, 0) node[circle, fill=black] (2) {}
        (4, 0) node[circle, fill=black] (3) {}
        (6, 0) node[circle, fill=black] (4) {}
        (5, -1) node[circle, fill=black] (5) {};

        \draw (1) -- (2);
        \draw (2) -- (3);
        \draw (3) -- (4);
        \draw (3) -- (5);
        \draw (4) -- (5)
        }
\end{proof}