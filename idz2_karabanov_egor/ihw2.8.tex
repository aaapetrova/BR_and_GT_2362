\begin{proof} $ $\\
Изначальный граф:
\begin{center}
\tikz {
        \path
(0, -7) node (S) {S}
(2, 0) node (a) {a}
(8, 0) node (1) {$\gamma$}
(2, -2) node (g) {g}
(8, -2) node (2) {$\delta$}
(2, -4) node (h) {h}
(8, -4) node (3) {$\varepsilon$}
(2, -6) node (b) {b}
(8, -6) node (4) {$\beta$}
(2, -8) node (c) {c}
(8, -8) node (5) {$\theta$}
(2, -10) node (d) {d}
(8, -10) node (6) {$\zeta$}
(2, -12) node (e) {e}
(8, -12) node (7) {$\alpha$}
(2, -14) node (f) {f}
(8, -14) node (8) {$\eta$}
(10, -7) node (F) {F};
\draw[->] (S) -- (a);
\draw[->] (S) -- (g);
\draw[->] (S) -- (h);
\draw[->] (S) -- (b);
\draw[->] (S) -- (c);
\draw[->] (S) -- (d);
\draw[->] (S) -- (e);
\draw[->] (S) -- (f);

\draw[<-] (F) -- (1);
\draw[<-] (F) -- (2);
\draw[<-] (F) -- (3);
\draw[<-] (F) -- (4);
\draw[<-] (F) -- (5);
\draw[<-] (F) -- (6);
\draw[<-] (F) -- (7);
\draw[<-] (F) -- (8);

\draw[->] (a) -- (1);
\draw[->] (a) -- (2);
\draw[->] (a) -- (3);
\draw[->] (g) -- (1);
\draw[->] (g) -- (2);
\draw[->] (h) -- (1);
\draw[->] (b) -- (4);
\draw[->] (b) -- (5);
\draw[->] (c) -- (5);
\draw[->] (d) -- (2);
\draw[->] (d) -- (5);
\draw[->] (d) -- (6);
\draw[->] (e) -- (4);
\draw[->] (e) -- (7);
\draw[->] (e) -- (8);
\draw[->] (f) -- (6);
}
\end{center} $ $\\
Пройденные пути:\\
\{S, a, $\gamma$, F\}, \{S, g, $\delta$, F\}, \{S, h, $\gamma$, a, $\varepsilon$, F\}, \{S, b, $\beta$, F\}, \{S, c, $\theta$, F\}, \{S, d, $\zeta$, F\}, \{S, e, $\alpha$, F\}\\
Граф после применения алгоритма:
\begin{center}
% \usetikzlibrary{graphs,automata,positioning}
\tikz {
        \path
(0, -7) node (S) {S}
(2, 0) node (a) {a}
(8, 0) node (1) {$\gamma$}
(2, -2) node (g) {g}
(8, -2) node (2) {$\delta$}
(2, -4) node (h) {h}
(8, -4) node (3) {$\varepsilon$}
(2, -6) node (b) {b}
(8, -6) node (4) {$\beta$}
(2, -8) node (c) {c}
(8, -8) node (5) {$\theta$}
(2, -10) node (d) {d}
(8, -10) node (6) {$\zeta$}
(2, -12) node (e) {e}
(8, -12) node (7) {$\alpha$}
(2, -14) node (f) {f}
(8, -14) node (8) {$\eta$}
(10, -7) node (F) {F};

\draw[<-] (S) -- (f);
\draw[<-] (F) -- (8);

\draw[->] (a) -- (1);
\draw[->] (a) -- (2);
\draw[<-, red] (a) -- (3);
\draw[->] (g) -- (1);
\draw[<-, red] (g) -- (2);
\draw[<-, red] (h) -- (1);
\draw[<-, red] (b) -- (4);
\draw[->] (b) -- (5);
\draw[<-, red] (c) -- (5);
\draw[->] (d) -- (2);
\draw[->] (d) -- (5);
\draw[<-, red] (d) -- (6);
\draw[->] (e) -- (4);
\draw[<-, red] (e) -- (7);
\draw[->] (e) -- (8);
\draw[->] (f) -- (6);
}
\end{center} $ $\\
\textbf{Ответ:}\ $\tilde{E} = \{a\varepsilon,  g\delta,  h\gamma,  b\beta,  c\theta,  d\zeta,  e\alpha\}$
\end{proof}